\chapter{ENADE 2011}

\section{\color{blue} Quest\~oes}

\subsection{\color{blue} Quest\~ao 1}

Em um pr\'edio de 8 andares, 5 pessoas aguardam o elevador no andar t\'erreo. Considere que elas entrar\~ao no elevador e sair\~ao, de maneira aleat\'oria, nos andares de 1 a 8.

Com base nessa situa\c c\~ao, fa\c ca o que se pede nos itens a seguir, apresentando o procedimento de c\'alculo utilizado na sua resolu\c c\~ao.

\begin{enumerate}

\item[(a)] Calcule a probabilidade de essas pessoas descerem em andares diferentes. (valor: 6,0 pontos).

\item[(b)] Calcule a probabilidade de duas ou mais pessoas descerem em um mesmo andar. (valor: 4,0 pontos).

\end{enumerate}

\subsection{\color{blue} Quest\~ao 2}

Considere a sequ\^encia num\'erica definida por $$\cases{a_1 = a;&\cr
a_{n+1}= \displaystyle{4a_n\over2+a_n^2}, & para $n\geq 1.$\cr}$$ 

Use o princ\'\i pio de indu\c c\~ao finita e mostre que $a_n<\sqrt{2}$, para todo n\'umero natural $n\geq 1$ e para $0<a<\sqrt{2}$, seguindo os passos indicados nos itens a seguir:

\begin{enumerate}

\item[(a)] escreva a hip\'otese e a tese da propriedade a ser demonstrada; (valor: 1,0 ponto)

\item[(b)] mostre que $\displaystyle{s = \frac{4a}{2+a^2}>0}$, para todo $a>0$; (valor: 1,0 ponto)

\item[(c)] prove que $s^2<2$, para todo $0<a<\sqrt{2}$; (valor: 3,0 pontos)

\item[(d)] mostre que $0<s<\sqrt{2}$; (valor: 2,0 pontos)

\item[(e)] suponha que $a_n<\sqrt{2}$ e prove que $a_{n+1}<2$; (valor: 1,0 ponto)

\item[(f)] conclua a prova por indu\c c\~ao. (valor: 2,0 pontos)

\end{enumerate}

\subsection{\color{blue} Quest\~ao 3}

O Teorema do Valor Intermedi\'ario \'e uma proposi\c c\~ao muito importante da an\'alise matem\'atica, com in\'umeras aplica\c c\~oes te\'oricas e pr\'aticas. Uma demonstra\c c\~ao anal\'\i tica desse teorema foi feita pelo matem\'atico Bernard Bolzano [1781 – 1848]. Nesse contexto, fa\c ca o que se pede nos itens a seguir:

\begin{enumerate}

\item[(a)] Enuncie o Teorema do Valor Intermedi\'ario para fun\c c\~oes reais de uma vari\'avel real; (valor: 2,0 pontos)

\item[(b)] Resolva a seguinte situa\c c\~ao-problema.

O vencedor da corrida de S\~ao Silvestre-2010 foi o brasileiro Mailson Gomes dos Santos, que fez o percurso de 15 km em 44 min e 7 seg. Prove que, em pelo menos dois momentos distintos da corrida, a velocidade instant\^anea de Mailson era de 5 metros por segundo. (valor: 4,0 pontos)

\item[(c)] Descreva uma situa\c c\~ao real que pode ser modelada por meio de uma fun\c c\~ao cont\'\i nua $f$, definida em um intervalo $[a , b]$, relacionando duas grandezas $x$ e $y$, tal que existe $k\in (a , b)$ com $f(x) \neq f(k)$, para todo $x\in (a , b), x \neq k$. Justifique sua resposta. (valor: 4,0 pontos)

\end{enumerate}

\section{\color{red} Solu\c c\~oes}

\subsection{\color{red} Quest\~ao 1}

\begin{enumerate}

\item[(a)] Considerando que as pessoas escolhem de forma aleat\'oria o andar que desejam ir, cada uma das pessoas t\^em 8 possibilidades, totalizando, pelo {\it princ\'\i pio multiplicativo} $8^5$ situa\c c\~oes diferentes, mas as que todas as pessoas saem em andares diferentes ocorrem do seguinte modo: a primeira tem 8 escolhas, a segunda apenas 7, pois n\~ao pode sair no mesmo andar da primeira, a terceira 6, a quarta 5 e a quinta 4, ou seja s\~ao $8.7.6.5.4$ casos favor\'aveis. Portanto a probabilidade deles ocorrerem \'e $$P_1= \frac{8.7.6.5.4}{8^5}=\frac{7.6.5.4}{8^4}=\frac{7.5.3}{8^3}=\frac{105}{512}$$

\item[(b)] A probabilidade de mais de uma pessoa descerem num mesmo andar \'e a probabilidade complementar do item anterior, ou seja: $$P_2=1-P_1=1-\frac{105}{512}=\frac{407}{512}$$

\end{enumerate}

\subsection{\color{red} Quest\~ao 2}

\begin{enumerate}

\item[(a)] Hip\'otese do {\it Princ\'\i pio da Indu\c c\~ao}: $a_1=a$; $\displaystyle a_{n+1}  = \frac{4a_n}{2+a_n^2}, \rm{para } n\geq 1$ e $0<a<\sqrt{2}$ e a tese \'e: $a_n<\sqrt{2}, \forall n\geq 1$

\item[(b)] Se $\displaystyle s=\frac{4a}{2+a^2}$ e pela hip\'otese de indu\c c\~ao $a>0$, ent\~ao $4a>0$ e $2+a^2>0$, portanto $s>0$

\item[(c)] Como $\displaystyle s=\frac{4a}{2+a^2}$ temos que: $$s^2=\frac{16a^2}{(2+a^2)^2}=\frac{16a^2}{4+4a^2+a^4}=\frac{16a^2}{(a^2-2)^2+8a^2}<\frac{16a^2}{8a^2}=2$$ portanto provamos que $s^2<2$.

\item[(d)] Temos que $s$ \'e sempre positiva e $0<s^2<2$, portanto se extrairmos a raiz quadrada obtemos: $0<s<\sqrt{2}$

\item[(e)] Como temos $a_n<\sqrt 2$ e $s=\displaystyle \frac{4a}{2+a^2}<\sqrt 2, \forall a, a<\sqrt 2$, logo: $a_{n+1}=\displaystyle \frac{4a_n}{2+a_n^2}<\sqrt 2<2$

\item[(f)] Para $n=1$ temos: $a_2=s<\sqrt 2$, \'e valida a hip\'otese. E como foi mostrado no item anterior: $a_{n+1}=\displaystyle \frac{4a_n}{2+a_n^2}<\sqrt 2$, assim conclu\'\i mos a indu\c c\~ao.

\end{enumerate}

\subsection{\color{red} Quest\~ao 3}

\begin{enumerate}

\item[(a)] Se $f$ \'e uma fun\c c\~ao cont\'\i nua em um intervalo $[a,b]$, ent\~ao o {\it Teorema do Valor Intermedi\'ario} diz que para todo $f(a)\leq k \leq f(b)$ existe um n\'umero $c\in (a,b)$ tal que: $f(k)=c$. 

\item[(b)] Considerando que a velocidade do corredor brasileiro possa ser expressa por uma fun\c c\~ao cont\'\i nua, $15 (km)= 15000 (m)$ e como ele percorreu este percurso em $44 (min)= 2640 (s)$ e $7 (seg)$, ou seja $2647 (seg)$, sua velocidade m\'edia foi $\displaystyle v_m= \frac{15000}{2647}\approx 5,6 (m/s)$. Como os corredores iniciam a corrida parados, temos que $v_0=0$ e considerando que ele tenha parado no instante que terminou a corrida, temos $v_{2647}=0$. Pelo teorema enunciado existe um \'unico momento $t$ em que $v_t=5,6 (m/s)$, mas como $5<5,6$ e $v_0=v_{2647}=0$, ent\~ao existem pelo menos dois instantes $a$ e $b$, por exemplo, em que a velocidade foi $5 (m/s)$.

\item[(c)] Qualquer situa\c c\~ao problema que pode ser modelada por uma fun\c c\~ao injetora.

\end{enumerate}

