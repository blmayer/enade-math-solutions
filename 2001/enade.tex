\chapter{ENADE 2001}

\section{\color{blue} Quest\~oes}

\subsection{\color{blue} Quest\~ao 1}

Sabendo-se que para todo n\'umero real $\theta$ tem-se que $e^{i\theta}= \cos (\theta) + i \sin (\theta)$, deduza as f\'ormulas

\begin{enumerate}

\item[(a)] $\sin (\alpha + \beta) = \sin (\alpha) \cos (\beta) + \cos (\alpha) \sin (\alpha)$ (valor: 10,0 pontos)

\item[(b)] $\cos (\alpha + \beta) = \cos (\alpha) \cos (\beta) - \sin (\alpha) \sin (\beta)$ (valor: 10,0 pontos)

\end{enumerate}

\subsection{\color{blue} Quest\~ao 2}

Uma piscina, vazia no instante $t = 0$, \'e abastecida por uma bomba d’\'agua cuja vaz\~ao no instante $t$ (horas) \'e $V(t)$ (metros c\'ubicos por hora).

\begin{enumerate}

\item[(a)] Determine o volume da piscina sabendo que, se $V(t) = 500$, a piscina fica cheia em $5$ horas. (valor: 5,0 pontos)

\item[(b)] Determine em quanto tempo a piscina ficaria cheia se $V(t) = 50 t$. (valor: 15,0 pontos)

\end{enumerate}

\subsection{\color{blue} Quest\~ao 3}

Sejam $A$ uma matriz real $2 \times 2$ com autovalores $\frac1{2}$ e $\frac1{3}$ e $\textbf v$ um vetor de $\mathbb R^2$.

\begin{enumerate}

\item[(a)] $A$ \'e diagonaliz\'avel? Justifique sua resposta. (valor: 5,0 pontos)

\item[(b)] Considere a seqü\^encia $\textbf v , A\textbf v , A^2\textbf v , A^3\textbf v , ... , A^n\textbf v , ... $. Prove que essa seqü\^encia \'e convergente. (valor: 15,0 pontos)

\end{enumerate}

\subsection{\color{blue} Quest\~ao 4}

Sejam $X$ e $Y$ espa\c cos m\'etricos, $A \subset X$ e $f: X \to Y$ uma fun\c c\~ao.

\begin{enumerate}

\item[(a)] Qual \'e o significado de “$A$ \'e aberto”? (valor: 5,0 pontos)

\item[(b)] Qual \'e o significado de “$A$ \'e fechado”? (valor: 5,0 pontos)

\item[(c)] Qual \'e o significado de “$f$ \'e cont\'\i nua em $X$”? (valor: 5,0 pontos)

\item[(d)] Se $a \in Y$ e $f$ \'e cont\'\i nua em $X$, mostre que o conjunto solu\c c\~ao da equa\c c\~ao $f(x) = a$ \'e fechado. (valor: 5,0 pontos)

\end{enumerate}

\subsection{\color{blue} Quest\~ao 5}

O corpo $\mathbb Z_2$ dos inteiros m\'odulo $2$ \'e formado por dois elementos, $0$ e $1$, com as opera\c c\~oes usuais de adi\c c\~ao e multiplica\c c\~ao definidas pelas t\'abuas abaixo.

$$\begin{array}{|c|c|c|}
\hline + & 0 & 1 \\
\hline 0 & 0 & 1 \\
\hline 1 & 1 & 0 \\
\hline
\end{array} \qquad \qquad
\begin{array}{|c|c|c|}
\hline \times & 0 & 1 \\
\hline 0 & 0 & 0 \\
\hline 1 & 0 & 1 \\
\hline
\end{array}$$

Considere em $\mathbb Z_2[x]$ – isto \'e, no anel dos polinômios na indeterminada $x$ cujos coeficientes pertencem a $\mathbb Z_2$ –, o polinômio de grau $2$, $q (x) = x^2 + x + 1$.

\begin{enumerate}

\item[(a)] Mostre que $q(x)$ n\~ao tem ra\'\i zes em $\mathbb Z_2$. (valor: 5,0 pontos)

\item[(b)] $q(x)$ sendo irredut\'\i vel, sabe-se, pelo Teorema de Kronecker, que existem um corpo $E$, que \'e uma extens\~ao de $\mathbb Z_2$ (ou seja, tal que $\mathbb Z_2$ \'e um subcorpo de $E$) e um elemento $\alpha \in E$ tal que $\alpha \notin \mathbb Z_2$ e $q(\alpha) = 0$. Determine o n\'umero m\'\i nimo de elementos que $E$ pode ter e construa as t\'abuas de adi\c c\~ao e de multiplica\c c\~ao em $E$. (valor: 15,0 pontos)

\end{enumerate}

\section{\color{red} Solu\c c\~oes}


\subsection{\color{red} Quest\~ao 2}

\begin{enumerate}

\item[(a)] $V=500\times 5=2500$ (metros c\'ubicos)

\item[(b)] $2500=\displaystyle \int_0^{t_1} 50 t dt \longrightarrow 2500=\frac{50t_1^2}{2}$, ou seja: $t_1=10$ (horas).

\end{enumerate}

\subsection{\color{red} Quest\~ao 3}

\begin{enumerate}

\item[(a)] $q(0)=1\neq 0$ e $q(1)=3=1\neq 0$ portanto \'e irredut\'\i vel.

\item[(b)] o Conjunto $E$ que se diz \'e $\mathbb Z_2$ extendido com as ra\'\i zes de $q(x)$, ou seja $\alpha=\displaystyle\frac{-1\pm \sqrt{1-4}}{2}$

\end{enumerate}

