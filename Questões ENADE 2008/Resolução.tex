\documentclass[12pt]{article}

\usepackage[brazilian]{babel}
\usepackage[utf8]{inputenc}
\usepackage[T1]{fontenc}
\usepackage{lmodern}
\usepackage{textcomp}
\usepackage{xcolor}
\usepackage{amsmath}
\usepackage{amssymb}

\title{\hrule \vspace{11pt} \Large{\color{red} UNIVERSIDADE PRESBITERIANA MACKENZIE} \vspace{10pt}\\
\hrule \vspace{60pt}
\color{blue} Resolução de Questões do ENADE}
\author{Brian Mayer\\
\color{red} Matemática - 8º Semestre}
\date{}

\begin{document}

\maketitle

\begin{abstract}
Neste documento será resolvida a questão discursiva de Bacharelado em Matemática do ENADE 2008 (Exame NAcional de Desempenho dos Estudantes) aplicada pelo SINAES (SIstema Nacional de Avaliação da Educação Superior) para apresentação ao Prof. Dr. Ariovaldo como requisito para obtenção de nota na disciplina de Seminários de Matemática II e também para o interesse geral no desenvolvimento e treinamento matemático empregado neste trabalho. O texto da questão não foi modificado, apenas rescrito e reformatado devido ao \emph{software} utilizado neste documento, i.e. \TeX. A questão aborda o tema de Análise Matemática, no que diz respeito à diferenciação. A solução contida neste trabalho será apresentada na lousa usando este documento apenas como um guia, e é resultado da mistura entre a criatividade do autor e de uma pesquisa de \emph{internet}.
\end{abstract}

\newpage

\section*{\color{blue} Questões}

\subsection*{\color{blue} Questão 1}

Considere uma função derivável $f: \mathbb{R} \to \mathbb{R}$ que satisfaz à seguinte condição:

Para qualquer número real $k\neq 0$, a função $g_k (x)$ definida por $g_k (x)=x-kf(x)$ não é injetora.

Com base nessa propriedade, faça o que se pede nos itens a seguir e transcreva suas respostas para o Caderno de Respostas, nos locais devidamente indicados.

\begin{description}

\item[a)] Mostre que, se $g'_k(x_0)=0$ para algum $k\neq 0$, então $f' (x_0)=\frac1{k}$ (valor: 3,0 pontos).

\item[b)] Mostre que, para cada $k \in \mathbb{R}$ não-nulo, existem números $\alpha_k$ e $\beta_k$ tais que $g_k(\alpha_k) = g_k(\beta_k)$. Além disso, justifique que, para todo $k \in \mathbb{R}$ não-nulo, existe um número $\theta_k$ tal que $g'_k(\theta_k)=0$. (valor: 3,0 pontos).

\item[c)] Mostre que a função derivada de primeira ordem $f'$ não é limitada. (valor: 4,0 pontos).


\end{description}

\section*{\color{red} Soluções}

\subsection*{\color{red} Questão 1}

\begin{description}

\item[a)] Derivando a função definida no item (a): $g'_k(x)=1-kf'(x)$. Fazendo $g'_k(x)=0$ temos: $0=1-kf'(x)$, ou seja: $f'(x_0)=\frac1{k}$, para um certo $x_0$

\item[b)] Como o exercício nos diz que a função $g_k(x)$ não é injetora, essa definição implica que existem $\alpha$ e $\beta$, diferentes, tais que: $g_k(\alpha)=g_k(\beta)$, mas como a mudança do valor de $k$ gera novas funções injetoras, é cômodo escrever $g_k(\alpha_k)=g_k(\beta_k)$ para mostrar tal fato; Usando o resultado do item (a), temos que se $g'_k(\theta_k)=0$ então $f'(\theta_k)=\frac1{k}$, portanto para cada valor de $k\neq 0$ temos uma função $g'_k(\theta_k)=0$

\item[c)] A função $f'$ não é limitada pois a função $1/x$ , para $x \neq 0$ não é limitada.

\end{description}

\end{document}

