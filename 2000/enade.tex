\chapter{ENADE 2000}

\section{\color{blue} Quest\~oes}

\subsection{\color{blue} Quest\~ao 1}

Seja $\gamma$ um caminho no plano complexo, fechado, simples, suave (isto \'e, continuamente deriv\'avel) e que n\~ao passa por $i$ nem por $-i$. Quais s\~ao os poss\'\i veis valores da integral $\displaystyle \int_\gamma \frac{dz}{1+z^2}$? (valor: 20,0 pontos)

\subsection{\color{blue} Quest\~ao 2}

Uma fun\c c\~ao $u: \mathbb R^2 \to \mathbb R$, com derivadas cont\'\i nuas at\'e a 2\textordfeminine\ ordem, \'e dita harmônica em $\mathbb R^2$ se satisfaz a Equa\c c\~ao de Laplace: $$\Delta u={\partial^2u\over\partial x^2}+{\partial^2u\over\partial y^2}=0 \qquad \rm{em}\;\; \mathbb R^2$$ Mostre que se $u$ e $u^2$ s\~ao harmônicas em $\mathbb R^2$, ent\~ao $u$ \'e uma fun\c c\~ao constante. (valor: 20,0 pontos)

\subsection{\color{blue} Quest\~ao 3}

Seja $\{A_n\}, n\in \mathbb N$, uma seqü\^encia de n\'umeros reais positivos e considere a s\'erie de fun\c c\~oes de uma vari\'avel real $t$ dada por $\displaystyle \sum_{n=0}^\infty (A_n)^t$. Suponha que tal s\'erie converge se $t = t_0 \in \mathbb R$. Prove que ela converge uniformemente no intervalo $[t_0, \infty [$. (valor: 20,0 pontos)

\subsection{\color{blue} Quest\~ao 4}

Sejam $A =\left(\matrix{0 & -1 & 3 \cr 0 & 2 & 0 \cr 0 & -1 & 0}\right)$ e $n$ um inteiro positivo. Calcule $A^n$.

\paragraph{Sugest\~ao:} Use a Forma Canônica de Jordan ou o Teorema de Cayley-Hamilton. (valor: 20,0 pontos)

\section{\color{red} Solu\c c\~oes}

\subsection{\color{red} Quest\~ao 2}

Como $u$ e $u^2$ s\~ao fun\c c\~oes harmônicas, ent\~ao: $$\frac{\partial^2u}{\partial x^2}+\frac{\partial^2u}{\partial y^2}=0$$ e $$\frac{\partial^2}{\partial x^2}(u^2)+\frac{\partial^2}{\partial y^2}(u^2)=0$$ como $\frac{\partial}{\partial x}(u^2)=2u\frac{\partial u}{\partial x}$, derivando novamente: $\frac{\partial}{\partial x}\left(2u\frac{\partial u}{\partial x}\right)=2\left(\frac{\partial u}{\partial x}\right)^2+2u\frac{\partial^2 u}{\partial x^2}$ e o mesmo acontece com a vari\'avel $y$, desse modo nossa equa\c c\~ao de Laplace toma a forma: $$\left(\frac{\partial u}{\partial x}\right)^2+u{\partial^2 u\over\partial x^2}+\left(\frac{\partial u}{\partial y}\right)^2+u{\partial^2 u\over\partial y^2}=\left({\partial u\over\partial x}\right)^2+\left({\partial u\over\partial y}\right)^2=0$$ pois  $u$ \'e harm\^onica. Portanto $\displaystyle{\partial u\over\partial x}=0$ e $\displaystyle\frac{\partial u}{\partial y}=0$. Resolvendo estas duas equa\c c\~oes, temos: (i) $u(x,y)= c+\phi(y)$ e (ii) $u(x,y)=k+\psi(x)$, derivando a primeira em rela\c c\~ao a $y$ e a segunda em rela\c c\~ao a $x$, obtemos: $\phi'(y)=0$ e $\psi'(x)=0$ respectivamente, o que indica que estas fun\c c\~oes s\~ao constantes. Assim necessariamente $\phi(y)=k$ e $\psi(x)=c$ e temos a unica solu\c c\~ao: $u(x,y)=c+k=C$


