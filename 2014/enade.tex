\chapter{ENADE 2014}

\section{\color{blue} Quest\~oes}

\subsection{\color{blue} Quest\~ao 1}

Os principais efeitos visuais da computa\c c\~ao gr\'afica vistos em uma tela s\~ao resultados de aplica\c c\~oes de transforma\c c\~oes lineares. Transla\c c\~ao, rota\c c\~ao, redimensionamento e altera\c c\~ao de cores s\~ao apenas alguns exemplos.

Considere que uma tela \'e cortada por dois eixos, $x$ e $y$, ortogonais entre si, formando um sistema de coordenadas com origem no centro da tela. Suponha que, nessa tela plana, existe a imagem de uma elipse com eixo maior de tamanho 4, paralelo ao eixo $x$, e cujos focos t\^em coordenadas $(-1,2)$ e $(1,2)$. Considere $T$ um operador linear definido em $\mathbb R^2$.

De acordo com as informa\c c\~oes acima, fa\c ca o que se pede nos itens a seguir, apresentando os c\'alculos utilizados na sua resolu\c c\~ao.

\begin{enumerate}

\item[(a)] Mostre que o ponto $(0,2+\sqrt{3})$ pertence \`a elipse. (valor: 3,0 pontos)

\item[(b)] Suponha que, em cada ponto da tela, seja aplicado o operador linear $T(x,y)=(x+y,-2x+4y)$. Quais ser\~ao as coordenadas dos focos da elipse ap\'os a aplica\c c\~ao de $T$? (valor: 3,0 pontos)

\item[(c)] Calcule os autovalores do operador linear $T(x,y)=(x+y,-2x+4y)$. (valor: 4,0 pontos)

\end{enumerate}
\subsection{\color{blue} Quest\~ao 2}

O n\'umero de ouro \'e conhecido h\'a mais de dois mil anos, sendo encontrado nas artes, nas pir\^amides do Egito e na natureza. Para construir o n\'umero de ouro apenas com o aux\'ilio de uma r\'egua n\~ao graduada e de um compasso, utiliza-se o seguinte procedimento: dado um segmento $AB$ qualquer, marca-se o seu ponto m\'edio; constr\'oi-se o segmento $BC$ perpendicular a $AB$ e com a metade do comprimento de $AB$; marca-se o ponto $E$ sobre a hipotenusa do tri\^angulo $ABC$, tal que $\overline{EC}$ e $\overline{BC}$ sejam iguais; e determina-se o ponto $D$ no segmento $AB$ tal que $\overline{AD}$ e $\overline{AE}$ sejam iguais. Com esse procedimento, o ponto $D$ divide o segmento $AB$ na raz\~ao \'aurea.

A partir da constru\c c\~ao geom\'etrica do n\'umero de ouro e considerando $x$ como o comprimento do segmento $AB$, fa\c ca o que se pede nos itens a seguir, apresentando os c\'alculos utilizados na sua resolu\c c\~ao.

\begin{enumerate}

\item[(a)] Determine o comprimento do segmento $AC$ em fun\c c\~ao de $x$. (valor: 4,0 pontos)

\item[(b)] Determine o comprimento do segmento $AD$ em fun\c c\~ao de $x$. (valor: 4,0 pontos)

\item[(c)] Determine o n\'umero de ouro dado pelo quociente $\overline{AB}\over\overline{AD}$. (valor: 2,0 pontos)
\end{enumerate}

\subsection{\color{blue} Quest\~ao 3}

A Torre de Han\'oi foi inventada por Edouard Lucas em 1883. H\'a uma hist\'oria sobre a Torre, imaginada pelo pr\'oprio Lucas:

No come\c co dos tempos, Deus criou a Torre de Brahma, que cont\'em tr\^es pinos de diamante e colocou no primeiro pino 64 discos de ouro maci\c co. Deus, ent\~ao, chamou seus sacerdotes e ordenou-lhes que transferissem todos os discos para o terceiro pino, segundo certas regras. Os sacerdotes, ent\~ao, obedeceram e com\c caram o seu trabalho, dia e noite. Quando eles terminassem, a torre de Brahma iria ruir e o mundo acabaria.

\vglue 10 pt

\pdfximage width 100pt height 110pt {hanoi.pdf}

\centerline{\pdfrefximage\pdflastximage}

Esse \'e um dos quebra-cabe\c cas matem\'aticos mais populares, que consiste de $n$ discos com um furo em seu centro e de tamanhos diferentes e de uma base com tr\^es pinos na posi\c c\~ao vertical onde s\~ao colocados os discos. O jogo mais simples \'e constituido de tr\^es pinos mas e quantidade pode variar, deixando o jogo mais dif\'icil \`a medida que o n\'umero de discos aumenta. Os discos formam uma torre onde todos s\~ao colocados em um dos pinos em ordem decrescente de tamanho. O objetivo do quebra-cabe\c ca \'e transferir toda a torre de discos para um dos outros pinos, que est\~ao inicialmente vazios, de modo que cada movimento \'e feito somente com um disco, nunca havendo um disco maior sobre um disco menor, como mostra a figura acima.

Considerando uma torre de Han\'oi de 3 pinos, fa\c ca o que se pede nos itens a seguir.

\begin{enumerate}

\item[(a)] Ao planejar uma aula de matem\'atica utilizando-se a Torre de Han\'oi, quais seriam os objetivos a serem alcan\c cados de acordo com os Par\^ametros Curriculares Nacionais e o que se espera com o uso de jogos no processo de ensino-aprendizagem? (valor: 3,0 pontos)

\item[(b)] Cite tr\^es conceitos matem\'aticos de Educa\c c\~ao B\'asica que podem ser explorados em sala de aula utilizando-se a Torre de Han\'oi? (valor: 3,0 pontos)

\item[(c)] Obtenha uma f\'ormula para o n\'umero m\'inimo de movimentos necess\'arios para resolver a Torre de Han\'oi com discos. Justifique a sua resposta. (valor: 4,0 pontos)

\end{enumerate}

\subsection{\color{blue} Quest\~ao 4}

Atualmente, a maioria dos editores de texto oferece o recurso de corre\c c\~ao ortogr\'afica. Esse recurso consiste em destacas, entre as palavras digitadas, aquelas com poss\'\i veis erros de grafia. Por exemplo, quando se digita a palavra ``caza'', o recurso de corre\c c\~ao destaca essa palavra, pois a palavra ``caza'' n\~ao existe na l\'\i ngua portuguesa. Tamb\'em \'e comum o recurso de corre\c c\~ao ortogr\'afica sugerir uma outra palavra para substituir a palavra incorreta.

A sugest\~ao de quais palavras podem substituir a palavra incorreta \'e feita com uma medida da dist\^ancia entre a palavra incorreta e as palavras que constam no dicion\'ario do editor de texto. Existem diversas maneiras de medir a dist\^ancia entre duas palavras. Uma delas \'e a denominada {\it Dist\^ancia de Hamming}, na qual a medida da dist\^ancia entre duas palavras $x$ e $y$, em suas respectivas posi\c c\~oes. Mais formalmente, se $x=x_1x_2x_3\ldots x_n$ e $y=y_1y_2y_3\ldots y_n$ s\~ao palavras em que $x_i$ e $y_i$ s\~ao letras do alfabeto, para $i=1,2,3,\ldots,n$, ent\~ao $d(x,y)=\#(\{i:x_i\neq y_i$, com $i=1,2,3,\ldots n\})$, em que $\#(\{3\})=1$, j\'a que elas diferem apenas na terceira letra.

A partir dessas informa\c c\~oes, fa\c ca o que se pede nos itens a seguir.

\begin{enumerate}

\item[(a)] Mostre que a Dist\^ancia de Hamming \'e uma m\'etrica no conjunto das palavras com letras. (valor: 5,0 pontos)

\item[(b)] Mostre que o conjunto das palavras com letras, munido da Dist\^ancia de Hamming, \'e um espa\c co m\'etrico discreto. (valor: 5,0 pontos)

\end{enumerate}

\subsection{\color{blue} Quest\~ao 5}

Uma equa\c c\~ao diofantina linear nas inc\'ognitas $x$ e $y$ \'e uma equa\c c\~ao da forma $ax+by=c$, em que $a$, $b$ e $c$ s\~ao inteiros, e as \'unicas solu\c c\~oes $(x_0,y_0)$ que interessam s\~ao aquelas em que $x_0, y_0 \in \mathbb Z$.

Nesse contexto, considere que os ingressos de um cinema custam R\$ 9,00 para estudantes e R\$ 15,00 para o p\'ublico geral, e que, em certo dia, durante determinado per\'\i odo, a arrecada\c c\~ao nas bilheterias desse cinema foi R\$ 246,00.

A partir das informa\c c\~oes acima, fa\c ca o que se pede nos itens a seguir.

\begin{enumerate}

\item[(a)] Obtenha ema equa\c c\~ao diofantina linear que modele a situa\c c\~ao acima, indicando o significado das inc\'ognitas. (valor: 3,0 pontos)

\item[(b)] Quantas e quais s\~ao as solu\c c\~oes do problema descrito no item (a)? (valor: 7,0 pontos)

\end{enumerate}

\section{\color{red} Solu\c c\~oes}

