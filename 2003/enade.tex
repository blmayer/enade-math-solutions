\chapter{ENADE 2003}

\section{\color{blue} Quest\~oes}

\subsection{\color{blue} Quest\~ao 1}

Seja $I =\displaystyle \int_0 ^3\int_{\sqrt{\frac{x}{3}}}^1 e^{y^3} dydx$.

\begin{enumerate}

\item[(a)] Esboce graficamente a regi\~ao de integra\c c\~ao. (valor: 5,0 pontos)

\item[(b)] Inverta a ordem de integra\c c\~ao. (valor: 10,0 pontos)

\item[(c)] Calcule o valor de $I$. (valor: 5,0 pontos)

\end{enumerate}

\subsection{\color{blue} Quest\~ao 2}

Seja $\mathbb{Z}_{18}$ o anel dos inteiros m\'odulo 18 e seja $G$ o grupo multiplicativo dos elementos invert\'\i veis de $\mathbb{Z}_{18}$.

\begin{enumerate}

\item[(a)] Escreva todos os elementos do grupo $G$. (valor: 10,0 pontos)

\item[(b)] Mostre que $G$ \'e c\'\i clico, calculando explicitamente um gerador, ou seja, mostre que existe $g \in G$ tal que todos os elementos de $G$ s\~ao pot\^encias de $g$. (valor: 10,0 pontos)

\end{enumerate}

\subsection{\color{blue} Quest\~ao 3}

\begin{enumerate}

\item[(a)] Dada a matriz sim\'etrica $A = \matrix{ 1 & 6 \cr 6 & 4}$, escreva, em forma de polinômio $f(x,y)$, a forma quadr\'atica definida por $A$, isto \'e, calcule os coeficientes num\'ericos de
$f(x,y) = v^t A v$ , onde $v = \matrix{ x \cr y}$ e $v^t$ significa “$v$ transposto”. (valor: 5,0 pontos)

\item[(b)] Encontre uma matriz invert\'\i vel $P$ tal que $P^t A P = D$, onde $D$ \'e uma matriz diagonal. Para isto, basta tomar como $P$ uma matriz que tenha por colunas um par de autovetores ortonormais de $A$. (valor: 10,0 pontos)

\item[(c)] Na forma quadr\'atica $f(x,y) = v^t A v$, fa\c ca uma transforma\c c\~ao de coordenadas $v = P \tilde v$, sendo $\tilde v = \matrix{\tilde x \cr \tilde y}$, obtendo a forma quadr\'atica diagonalizada, isto \'e, sem o termo em $\tilde x \tilde y$ . (valor: 5,0 pontos)

\end{enumerate}

\subsection{\color{blue} Quest\~ao 4}

Seja $p(x) = x^n + a_{n-1} x^{n-1} + ... + a_1 x + a_0$ , com $n \geq 1$, um polinômio de coeficientes reais. Suponha que $p'(x)$ divide $p(x)$.

\begin{enumerate}

\item[(a)] Prove que o quociente $q(x) = \displaystyle \frac{p(x)}{p'(x)}$ \'e da forma $q(x) = \frac1{n} (x-x_0), x_0 \in \mathbb R$. (valor: 5,0 pontos)

\item[(b)] Encontre todos os polinômios $p(x)$ que satisfazem essa condi\c c\~ao, resolvendo a equa\c c\~ao diferencial $q(x) p'(x)-p(x) = 0$. (valor: 15,0 pontos)

\end{enumerate}

\subsection{\color{blue} Quest\~ao 5}

Dado um conjunto aberto $U \subset \mathbb R^3$ e um campo de vetores $X = (X_1 , X_2 , X_3 ): U \to \mathbb R^3$ diferenci\'avel, o divergente de $X$ \'e definido por $$div X=\frac{\partial X_1}{\partial x}+\frac{\partial X_2}{\partial y}+\frac{\partial X_3}{\partial z}$$

Para uma fun\c c\~ao de classe $C^2, f: U \to \mathbb R^3$ o laplaciano de $f$ \'e definido por $$\Delta f = \frac{\partial^2 f}{\partial x^2}+\frac{\partial^2 f}{\partial y^2}+\frac{\partial^2 f}{\partial z^2}$$

\begin{enumerate}

\item[(a)] Se $f: U \to \mathbb R$ \'e diferenci\'avel e $X: U \to \mathbb R^3$ \'e um campo de vetores diferenci\'avel, mostre que $$div(f X)=f\; div(X) + \nabla f \cdot X ,$$ sendo $\nabla f$ o gradiente de $f$ e $\nabla f \cdot X$ o produto interno entre $\nabla f$ e $X$. (valor: 5,0 pontos)

\item[(b)] Se $f: U \to \mathbb R$ \'e de classe $C^2$, mostre que $div(f\nabla f) = f \Delta f + || \nabla f ||^2$, sendo $||\;\;||$ a norma euclidiana. (valor: 5,0 pontos)

\item[(c)] Se $U = B = \{x \in \mathbb R^3: || x || < 1\}$ e $f: \bar B \to \mathbb R$ \'e de classe $C^3$ tal que $f(x) > 0$ para qualquer $x \neq 0$, $div(f \nabla f) = 5f$ e $|| \nabla f ||^2= 2f$, calcule $$ \int_S \frac{\partial f}{\partial N} dS ,$$ onde $\bar B$ \'e o fecho de $B$, $S$ \'e a fronteira de $B$, $N$ \'e a norma unit\'aria exterior a $S$, $\displaystyle \frac{\partial f}{\partial N}$ \'e a derivada direcional de $f$ na dire\c c\~ao de $N$ e $dS$ \'e o elemento de \'area de $S$. (valor: 10,0 pontos)

\end{enumerate}

\subsection{\color{blue} Quest\~ao 6}

Considere a fun\c c\~ao real $f$ definida, para $x \geq 0$, por $f(x) = \sqrt{2x}$.

\begin{enumerate}

\item[(a)] Prove que se $0 < x < 2$, ent\~ao $x < f(x) < 2$. (valor: 5,0 pontos)

\item[(b)] Prove que \'e convergente a seqü\^encia definida recursivamente por

\begin{enumerate}

\item[1.] $a_1=\sqrt 2$

\item[2.] $a_{n+1}=f(a_n)$, para todo $n \geq 1$

\end{enumerate}

(valor: 5,0 pontos)

\item[(c)] Calcule $\displaystyle \lim_{n \to \infty} a_n$ (valor: 10,0 pontos)

\end{enumerate}

\section{\color{red} Solu\c c\~oes}

\subsection{\color{red} Quest\~ao 1}

\begin{enumerate}

\item[(a)]

\begin{center}
\begin{picture}(200,150)
\put(0,20){\vector(1,0){200}}
\put(195,10){$x$}
\put(20,0){\vector(0,1){150}}
\put(9,145){$y$}
\put(100,52){\small $y=\sqrt{\frac{x}{3}}$}
\multiput(186,0)(0,11){14}{\line(0,1){5}}
\put(186,80){\circle*{3}}
\put(10,77){\small{$1$}}
\put(176,8){\small{$3$}}
\color{blue}
\qbezier(20,20)(20,50)(200,83)
\put(20,80){\line(1,0){164}}
\put(20,20){\line(0,1){60}}
\put(50,60){$R$}
\end{picture}
\end{center}

\item[(b)] $I =\displaystyle \int_0 ^3\int_{\sqrt{\frac{x}{3}}}^1 e^{y^3} dydx=\displaystyle \int_0 ^1\int_0^{3y^2} e^{y^3} dxdy$

\item[(c)] $I=\displaystyle \int_0 ^1\int_0^{3y^2} e^{y^3} dxdy=\int_0^13y^2e^{y^3}dy=e-1$ 

\end{enumerate}

\subsection{\color{red} Quest\~ao 2}

\begin{enumerate}

\item[(a)] $G=\{ 1,5,7,11,13 \}$

\end{enumerate}

\subsection{\color{red} Quest\~ao 4}

\begin{enumerate}

\item[(a)] Temos que $p'(x)=nx^{n-1}+a_{n-1}(n-1)x^{n-2}+...+a_1$. Como $p'(x)$ divide $p(x)$ ent\~ao $q(x)$ deve ser de grau um. Portando $q(x)=k(x-x_0)$.

O teorema fundamental da divis\~ao nos d\'a: $q(x)p'(x)=p(x)$, assim $k(x-x_0)[nx^{n-1}+a_{n-1}(n-1)x^{n-2}+...+a_1]=x^n + a_{n-1} x^{n-1} + ... + a_1 x + a_0$. Multiplicando o primeiro termo da esquerda temos $knx^n=x^n$, portanto $k=\frac1{n}$. Logo $q(x)= \frac1{n}(x-x_0)$.

\item[(b)] Multiplicando a equa\c c\~ao diferencial dada no enunciado por $\frac1{q(x)}$ obtemos: $$p'(x)-\frac{p(x)}{q(x)}=0$$ ent\~ao $\frac{p'}{p}=\frac{n}{x-x_0} \Longrightarrow \ln(p)=n\ln(x-x_0)+c$, portanto $p(x)=k(x-x_0)^n$.

\end{enumerate}

