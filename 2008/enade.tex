\chapter{ENADE 2008}

\section{\color{blue} Quest\~oes}

\subsection{\color{blue} Quest\~ao 1}

Considere uma fun\c c\~ao deriv\'avel $f: \mathbb{R} \to \mathbb{R}$ que satisfaz \`a seguinte condi\c c\~ao:

Para qualquer n\'umero real $k\neq 0$, a fun\c c\~ao $g_k (x)$ definida por $g_k (x)=x-kf(x)$ n\~ao \'e injetora.

Com base nessa propriedade, fa\c ca o que se pede nos itens a seguir e transcreva suas respostas para o Caderno de Respostas, nos locais devidamente indicados.

\begin{enumerate}

\item[(a)] Mostre que, se $g'_k(x_0)=0$ para algum $k\neq 0$, ent\~ao $f' (x_0)=\frac1{k}$ (valor: 3,0 pontos).

\item[(b)] Mostre que, para cada $k \in \mathbb{R}$ n\~ao-nulo, existem n\'umeros $\alpha_k$ e $\beta_k$ tais que $g_k(\alpha_k) = g_k(\beta_k)$. Al\'em disso, justifique que, para todo $k \in \mathbb{R}$ n\~ao-nulo, existe um n\'umero $\theta_k$ tal que $g'_k(\theta_k)=0$. (valor: 3,0 pontos).

\item[(c)] Mostre que a fun\c c\~ao derivada de primeira ordem $f'$ n\~ao \'e limitada. (valor: 4,0 pontos).

\end{enumerate}

\section{\color{red} Solu\c c\~oes}

\subsection{\color{red} Quest\~ao 1}

\begin{enumerate}

\item[(a)] Derivando a fun\c c\~ao definida no item (a): $g'_k(x)=1-kf'(x)$. Fazendo $g'_k(x)=0$ temos: $0=1-kf'(x)$, ou seja: $f'(x_0)=\frac1{k}$, para um certo $x_0$

\item[(b)] Como o exerc\'\i cio nos diz que a fun\c c\~ao $g_k(x)$ n\~ao \'e injetora, essa defini\c c\~ao implica que existem $\alpha$ e $\beta$, diferentes, tais que: $g_k(\alpha)=g_k(\beta)$, mas como a mudan\c ca do valor de $k$ gera novas fun\c c\~oes injetoras, \'e cômodo escrever $g_k(\alpha_k)=g_k(\beta_k)$ para mostrar tal fato; Usando o resultado do item (a), temos que se $g'_k(\theta_k)=0$ ent\~ao $f'(\theta_k)=\frac1{k}$, portanto para cada valor de $k\neq 0$ temos uma fun\c c\~ao $g'_k(\theta_k)=0$

\item[(c)] A fun\c c\~ao $f'$ n\~ao \'e limitada pois a fun\c c\~ao $1/x$ , para $x \neq 0$ n\~ao \'e limitada.

\end{enumerate}

