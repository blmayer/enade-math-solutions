\chapter{ENADE 2002}

\section{\color{blue} Quest\~oes}

\subsection{\color{blue} Quest\~ao 1}

Sejam $g$ e $h$ fun\c c\~oes deriv\'aveis de $\mathbb R$ em $\mathbb R$ tais que $g’(x)=h(x)$, $h’(x)=g(x)$, $g(0)=0$ e $h(0)=1$.

\begin{enumerate}

\item[(a)] Calcule a derivada de $h^2(x) - g^2(x)$. (valor: 10,0 pontos) 

\item[(a)] Mostre que $h^2(x) - g^2(x) = 1$, para todo $x$ em $\mathbb R$. (valor: 10,0 pontos)

\end{enumerate}

\subsection{\color{blue} Quest\~ao 2}

Em um espa\c co m\'etrico $M$, com dist\^ancia $d$, a bola aberta de raio $r > 0$ e centro $p\in M$ \'e o conjunto $B_r(p) = \{x\in M | d(x,p) < r\}$. Por defini\c c\~ao, um conjunto $A \subset M$ \'e aberto se para qualquer ponto $p \in A$ existir $\epsilon > 0$ tal que $B_\epsilon(p) \subset A$.

\begin{enumerate}

\item[(a)] Mostre que a uni\~ao de uma fam\'\i lia qualquer de conjuntos abertos \'e um conjunto aberto. (valor: 5,0 pontos)

\item[(b)] Mostre que a interse\c c\~ao de uma fam\'\i lia finita n\~ao vazia de conjuntos abertos \'e um conjunto aberto. (valor: 10,0 pontos) 

\item[(c)] Em $\mathbb R$, com a m\'etrica usual, o conjunto $\{0\}$ n\~ao \'e aberto. D\^e exemplo de uma fam\'\i lia infinita de conjuntos abertos de R cuja interse\c c\~ao seja $\{0\}$. (valor: 5,0 pontos)

\end{enumerate}

\subsection{\color{blue} Quest\~ao 3}

Seja $A$ uma matriz quadrada de ordem $n$.

\begin{enumerate}

\item[(a)] Defina autovalor de $A$. (valor: 5,0 pontos)

\item[(b)] Se $\lambda$ \'e um autovalor de $A$, mostre que $2\lambda$ \'e um autovalor de $2A$. (valor: 5,0 pontos)

\item[(c)] Se $\lambda$ \'e um autovalor de $A$, mostre que $\lambda ^2$ \'e um autovalor de $A^2$. (valor: 10,0 pontos)

\end{enumerate}

\subsection{\color{blue} Quest\~ao 4}

O complexo $w$ \'e tal que a equa\c c\~ao $z^2-wz+(1-i)=0$ admite $1+i$ como raiz. 

\begin{enumerate}

\item[(a)] Determine $w$. (valor: 5,0 pontos)

\item[(b)] Determine a outra raiz da equa\c c\~ao. (valor: 5,0 pontos)

\item[(c)] Calcule a integral $\displaystyle \int_\gamma \frac{dz}{z^2-wz+(1-i)}$, sendo $\gamma$ a circunfer\^encia descrita parametricamente por $\gamma(t)=\frac1{2} \cos(t)+i(\frac1{2}\sin(t)-1)$, $0\leq t \leq 2\pi$. (valor: 10,0 pontos)

\end{enumerate}

\subsection{\color{blue} Quest\~ao 5}

A s\'erie de pot\^encias a seguir define, no seu intervalo de converg\^encia, uma fun\c c\~ao $g$, $g(x) =\displaystyle 1-\frac{x^2}{2}+\frac{x^4}{4}-...+(-1)^n\frac{x^{2n}}{2n}+...$

\begin{enumerate}

\item[(a)] Determine o raio de converg\^encia $r$ da s\'erie. Justifique. (valor: 5,0 pontos) 

\item[(b)] Expresse $g'(x)$ como soma de uma s\'erie de pot\^encias, para $|x|<r$. (valor: 5,0 pontos)

\item[(c)] Expresse $g'(x)$, para $|x|<r$, em termos de fun\c c\~oes elementares (polinomiais, trigonom\'etricas, logar\'\i tmicas, exponenciais). (valor: 5,0 pontos)

\item[(d)] Expresse $g(x)$, para $|x|<r$, em termos de fun\c c\~oes elementares. (valor: 5,0 pontos)

\end{enumerate}

\subsection{\color{blue} Quest\~ao 6}

Uma fonte de luz localizada no ponto $L = (0,-1, 0)$ ilumina a superf\'\i cie dada, parametricamente, por $P(u,v) = (u + v, u^2, v)$.

\begin{enumerate}

\item[(a)] Calcule o vetor normal \`a superf\'\i cie, $\vec N (u,v)$, de forma que para $u = v = 0$ esse vetor seja $(0,-1, 0)$. (valor: 5,0 pontos)

\item[(b)] Trabalhando com os vetores $\vec N$ e $L-P$, d\^e uma condi\c c\~ao sobre $u$ e $v$ a fim de que o ponto $P(u,v)$ seja iluminado pela luz em $L$. (valor: 15,0 pontos)

\end{enumerate}

\section{\color{red} Solu\c c\~oes}

\subsection{\color{red} Quest\~ao 1}

\begin{enumerate}

\item[(a)] $(h^2(x)-g^2(x))'=2hh'-2gg'=2hg-2gh=0$

\item[(b)] Como $(h^2(x)-g^2(x))'=0$, temos que $h^2(x)-g^2(x)=k$. Mas $h(0)=1$ e $g(0)=0$ ent\~ao: $1^2-0^2=1\Longrightarrow k=1$

\end{enumerate}

\begin{enumerate}

\item[(a)] $(1+i)^2-w(1+i)+(1-i)=0\Longrightarrow w=1$

\item[(b)] Como o enunciado disse que $1+i$ \'e raiz, ent\~ao podemos rescrever a equa\c c\~ao dada. Assim: $z^2-z+(1-i)=0$, por inspe\c c\~ao percebemos que a outra raiz \'e $-i$, pois $(-i)^2-(-i)+(1-i)=-1+i+(1-i)=0$.

\item[(c)] $\displaystyle \int_\gamma \frac{dz}{z^2-wz+(1-i)}= \int_\gamma \frac{dz}{(z-1+i)(z+i)}$, decompondo com fra\c c\~oes parciais temos: $$\frac1{(z-1+i)(z+i)}=\frac{a+bi}{(z-1+i)}+\frac{c+di}{(z+i)}$$

Onde obtemos: $\displaystyle \frac{(a+bi+c+di)z+(-b-c-d)+(c-b+a-d)i}{(z-1+i)(z+i)}$.

A qual gera o seguinte sistema: $$\cases{a+bi+c+di&$=0$ \cr b+c+d&$=-1$ \cr c-b+a-d&$=0$\cr}$$

A primeira equa\c c\~ao nos d\'a que $a=-c$ e $b=-d$, substituindo na segunda equa\c c\~ao temos: $c=-1$, logo, $a=1$. Portanto as constantes s\~ao: $a=1$, $c=-1$, nosso sistema agora \'e apenas a equa\c c\~ao $b+d=0$ Portanto podemos fazer $b=d=0$. Assim as fra\c c\~oes s\~ao $\displaystyle \frac1{z-1+i}-\frac1{z+i}$, e a integral original torna-se $\displaystyle \int_\gamma \frac1{z-(1-i)}-\frac1{z+i}dz$. A circunfer\^encia $\gamma$ pode ser rescrita como: $\gamma(t)=\frac1{2}[\cos(t)+i\sin(t)]-i$, desse modo percebemos que est\'a centrada no ponto $(0,-i)$ e possui raio $1/2$. Como a primeira raiz est\'a fora da curva, o valor da integral \'e zero. Ent\~ao: $$\int_\gamma \frac1{z-(1-i)}-\frac1{z+i}dz=-\int_\gamma\frac1{z+i}dz=-2i\pi$$

\end{enumerate}

\subsection{\color{red} Quest\~ao 5}

\begin{enumerate}

\item[(a)] Pelo teste da raz\~ao temos: $$\lim_{n\to\infty}\left|\frac{(-1)^{n+1}\displaystyle \frac{x^{2(n+1)}}{2(n+1)}}{(-1)^n\displaystyle \frac{x^{2n}}{2n}}\right|=\lim_{n\to\infty}\left|\frac{x^{2(n+1)}}{2(n+1)}\frac{2n}{x^{2n}}\right|.$$

Assim $\lim_{n\to\infty} \left|\frac{n}{n+1}x^2\right|$ onde percebemos que $|x^2|<1$, ou simplesmente $|x|<1$. Ent\~ao o raio de convergencia $r$ da s\'erie \'e: $r<1$.

\item[(b)] Derivando: $g'(x)=\displaystyle -x+x^3-x^5+...+(-1)^n x^{2n-1}+...$

\item[(c)] Considere a s\'erie de {\it Maclaurin} da fun\c c\~ao $y=\ln(x+1)$, i. e. : $$\ln(x+1)=x - \frac1{2} x^2 + \frac1{3}x^3 - \frac1{4}x^4 + \frac1{5}x^5-\ldots$$ se utilizarmos $x^2$ no lugar de $x$, ter\'\i amos:

\begin{eqnarray*}
{d\over dx}\ln(x^2+1)&=&{2x\over x^2+1}\\
{d^2\over dx^2}\ln(x^2+1)&=&{2(x^2+1)-4x^2\over(x^2+1)^2}\\
&=&{1-2x^2\over(x^2+1)^2}
\end{eqnarray*}

\end{enumerate}

