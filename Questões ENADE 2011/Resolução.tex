\documentclass[12pt]{article}

\usepackage[brazilian]{babel}
\usepackage[utf8]{inputenc}
\usepackage[T1]{fontenc}
\usepackage{lmodern}
\usepackage{textcomp}
\usepackage{xcolor}
\usepackage{amsmath}
\usepackage{amssymb}

\title{\hrule \vspace{11pt} \Large{\color{red} UNIVERSIDADE PRESBITERIANA MACKENZIE} \vspace{10pt}\\
\hrule \vspace{60pt}
\color{blue} Resolução de Questões do ENADE}
\author{Brian Mayer\\
\color{red} Matemática - 8º Semestre}
\date{}

\begin{document}

\maketitle

\begin{abstract}
Neste documento serão resolvidas as três questões discursivas de matemática do ENADE 2011 (Exame NAcional de Desempenho dos Estudantes) aplicada pelo SINAES (SIstema Nacional de Avaliação da Educação Superior) para apresentação ao Prof. Dr. Ariovaldo como requisito para obtenção de nota na disciplina de Seminários de Matemática II e também para o interesse geral no desenvolvimento e treinamento matemático empregado neste trabalho. Os textos das questões não foram modificados, apenas rescritos e reformatados devido ao \emph{software} utilizado neste documento, i.e. \TeX. A primeira questão aborda o tema de Estatística, no que diz respeito a probabilidades, as segunda e terceira questões abrangem a área de Análise Matemática, onde \emph{a priori} apresenta conjecturas sobre uma sequência, e \emph{a posteriori} aplica esse campo para resolver uma situação-problema na Corrida de São Silvestre de 2010. As soluções contidas neste trabalho serão apresentadas na lousa usando este documento apenas como um guia, e são resultados da mistura entre a criatividade do autor e de uma pesquisa de \emph{internet}.
\end{abstract}

\section*{\color{blue} Questões}

\subsection*{\color{blue} Questão 1}

Em um prédio de 8 andares, 5 pessoas aguardam o elevador no andar térreo. Considere que elas entrarão no elevador e sairão, de maneira aleatória, nos andares de 1 a 8.

Com base nessa situação, faça o que se pede nos itens a seguir, apresentando o procedimento de cálculo utilizado na sua resolução.

\begin{description}

\item[a)] Calcule a probabilidade de essas pessoas descerem em andares diferentes. (valor: 6,0 pontos).

\item[b)] Calcule a probabilidade de duas ou mais pessoas descerem em um mesmo andar. (valor: 4,0 pontos).

\end{description}

\subsection*{\color{blue} Questão 2}

Considere a sequência numérica definida por $$\begin{cases}
a_1 & = a,\\
a_{n+1} & = \frac{4a_n}{2+a_n^2}, \text{para } n\geq 1.
\end{cases}$$ 

Use o princípio de indução finita e mostre que $a_n<\sqrt{2}$, para todo número natural $n\geq 1$ e para $0<a<\sqrt{2}$, seguindo os passos indicados nos itens a seguir:

\begin{description}

\item[a)] escreva a hipótese e a tese da propriedade a ser demonstrada; (valor: 1,0 ponto)

\item[b)] mostre que $\displaystyle{s = \frac{4a}{2+a^2}>0}$, para todo $a>0$; (valor: 1,0 ponto)

\item[c)] prove que $s^2<2$, para todo $0<a<\sqrt{2}$; (valor: 3,0 pontos)

\item[d)] mostre que $0<s<\sqrt{2}$; (valor: 2,0 pontos)

\item[e)] suponha que $a_n<\sqrt{2}$ e prove que $a_{n+1}<2$; (valor: 1,0 ponto)

\item[f)] conclua a prova por indução. (valor: 2,0 pontos)

\end{description}

\subsection*{\color{blue} Questão 3}

O Teorema do Valor Intermediário é uma proposição muito importante da análise matemática, com inúmeras aplicações teóricas e práticas. Uma demonstração analítica desse teorema foi feita pelo matemático Bernard Bolzano [1781 – 1848]. Nesse contexto, faça o que se pede nos itens a seguir:

\begin{description}

\item[a)] Enuncie o Teorema do Valor Intermediário para funções reais de uma variável real; (valor: 2,0 pontos)

\item[b)] Resolva a seguinte situação-problema.

O vencedor da corrida de São Silvestre-2010 foi o brasileiro Mailson Gomes dos Santos, que fez o percurso de 15 km em 44 min e 7 seg. Prove que, em pelo menos dois momentos distintos da corrida, a velocidade instantânea de Mailson era de 5 metros por segundo. (valor: 4,0 pontos)

\item[c)] Descreva uma situação real que pode ser modelada por meio de uma função contínua $f$, definida em um intervalo $[a , b]$, relacionando duas grandezas $x$ e $y$, tal que existe $k\in (a , b)$ com $f(x) \neq f(k)$, para todo $x\in (a , b), x \neq k$. Justifique sua resposta. (valor: 4,0 pontos)

\end{description}

\section*{\color{red} Soluções}

\subsection*{\color{red} Questão 1}

\begin{description}

\item[a)] Considerando que as pessoas escolhem de forma aleatória o andar que desejam ir, cada uma das pessoas têm 8 possibilidades, totalizando, pelo \emph{princípio multiplicativo} $8^5$ situações diferentes, mas as que todas as pessoas saem em andares diferentes ocorrem do seguinte modo: a primeira tem 8 escolhas, a segunda apenas 7, pois não pode sair no mesmo andar da primeira, a terceira 6, a quarta 5 e a quinta 4, ou seja são $8.7.6.5.4$ casos favoráveis. Portanto a probabilidade deles ocorrerem é $$P_1= \frac{8.7.6.5.4}{8^5}=\frac{7.6.5.4}{8^4}=\frac{7.5.3}{8^3}=\frac{105}{512}$$

\item[b)] A probabilidade de mais de uma pessoa descerem num mesmo andar é a probabilidade complementar do item anterior, ou seja: $$P_2=1-P_1=1-\frac{105}{512}=\frac{407}{512}$$

\end{description}

\subsection*{\color{red} Questão 2}

\begin{description}

\item[a)] Hipótese do \emph{Princípio da Indução}: $a_1=a$; $\displaystyle a_{n+1}  = \frac{4a_n}{2+a_n^2}, \text{para } n\geq 1$ e $0<a<\sqrt{2}$ e a tese é: $a_n<\sqrt{2}, \forall n\geq 1$

\item[b)] Se $\displaystyle s=\frac{4a}{2+a^2}$ e pela hipótese de indução $a>0$, então $4a>0$ e $2+a^2>0$, portanto $s>0$

\item[c)] Como $\displaystyle s=\frac{4a}{2+a^2}$ temos que: $$s^2=\frac{16a^2}{(2+a^2)^2}=\frac{16a^2}{4+4a^2+a^4}=\frac{16a^2}{(a^2-2)^2+8a^2}<\frac{16a^2}{8a^2}=2$$ portanto provamos que $s^2<2$.

\item[d)] Temos que $s$ é sempre positiva e $0<s^2<2$, portanto se extrairmos a raiz quadrada obtemos: $0<s<\sqrt{2}$

\item[e)] Como temos $a_n<\sqrt 2$ e $s=\displaystyle \frac{4a}{2+a^2}<\sqrt 2, \forall a, a<\sqrt 2$, logo: $a_{n+1}=\displaystyle \frac{4a_n}{2+a_n^2}<\sqrt 2<2$

\item[f)] Para $n=1$ temos: $a_2=s<\sqrt 2$, é valida a hipótese. E como foi mostrado no item anterior: $a_{n+1}=\displaystyle \frac{4a_n}{2+a_n^2}<\sqrt 2$, assim concluímos a indução.

\end{description}

\subsection*{\color{red} Questão 3}

\begin{description}

\item[a)] Se $f$ é uma função contínua em um intervalo $[a,b]$, então o \emph{Teorema do Valor Intermediário} diz que para todo $f(a)\leq k \leq f(b)$ existe um número $c\in (a,b)$ tal que: $f(k)=c$. 

\item[b)] Considerando que a velocidade do corredor brasileiro possa ser expressa por uma função contínua, $15 (km)= 15000 (m)$ e como ele percorreu este percurso em $44 (min)= 2640 (s)$ e $7 (seg)$, ou seja $2647 (seg)$, sua velocidade média foi $\displaystyle v_m= \frac{15000}{2647}\approx 5,6 (m/s)$. Como os corredores iniciam a corrida parados, temos que $v_0=0$ e considerando que ele tenha parado no instante que terminou a corrida, temos $v_{2647}=0$. Pelo teorema enunciado existe um único momento $t$ em que $v_t=5,6 (m/s)$, mas como $5<5,6$ e $v_0=v_{2647}=0$, então existem pelo menos dois instantes $a$ e $b$, por exemplo, em que a velocidade foi $5 (m/s)$.

\item[c)] Qualquer situação problema que pode ser modelada por uma função injetora.

\end{description}

\end{document}

