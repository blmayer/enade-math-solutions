\documentclass{report}

\usepackage{xcolor,amssymb}
\usepackage[brazilian]{babel}

\title{\Large{\color{red} UNIVERSIDADE PRESBITERIANA MACKENZIE} \\[30pt]
\color{blue} Resolu\c c\~ao das Quest\~oes Discursivas dos ENADEs}
\author{\sc Brian Mayer\\[10pt]
\color{red} Matem\'atica - 8\textordmasculine\ Semestre}

\setlength{\parskip}{5pt plus3pt minus3pt}

\begin{document}

\maketitle

\begin{abstract}

Neste documento ser\~ao resolvidas as quest\~oes discursivas das provas de matem\'atica a niveis de licenciatura e bacharelado do ENADE (Exame NAcional de Desempenho dos Estudantes) aplicadas pelo SINAES (SIstema Nacional de Avalia\c c\~ao da Educa\c c\~ao Superior) dos anos de 1998 at\'e 2014, a \'ultima prova aplicada at\'e o presente, para apresenta\c c\~ao semanal ao Prof. Dr. Ariovaldo Jos\'e de Almeida como requisito para obten\c c\~ao de nota na disciplina de Semin\'arios de Matem\'atica II e tamb\'em para o interesse geral no desenvolvimento e treinamento matem\'atico empregado neste trabalho. Os textos das quest\~oes n\~ao foram modificados, apenas rescritos e reformatados devido ao {\it software} utilizado neste documento, i.e. \LaTeXe. As solu\c c\~oes contidas neste trabalho ser\~ao apresentadas na lousa usando este documento apenas como um guia, e s\~ao resultados da mistura entre a criatividade do autor e de uma pesquisa de {\it internet}. 

\end{abstract}

\tableofcontents

\part{Considera\c c\~oes Iniciais}

\chapter{Introdu\c c\~ao}

Come\c cando com o ENADE de 1998 que possui cinco (5) quest\~oes discursivas, aborda os temas de c\'alculo de \'areas, solu\c c\~oes de equa\c c\~oes diferenciais, demonstra\c c\~oes a respeito de converg\^encia de sequ\^encias, integrais complexas e opera\c c\~oes com an\'eis e corpos. No ENADE de 1999 temos cinco (5) quest\~oes, a primeira fazendo uma aplica\c c\~ao de equa\c c\~oes diferenciais no crescimento de uma popula\c c\~ao, e pedindo uma solu\c c\~ao anal\'\i tica para a mesma, outra quest\~ao na \'area de c\'alculo pede o valor de uma integral complexa em um curva muito conhecida pela matem\'atica, as demais perguntas s\~ao de \'algebra e an\'alise, onde os problemas de \'algebra se concentraram no t\'opico de polinômios, na an\'alise s\~ao abordados sequ\^encias, fun\c c\~oes e campos vetoriais. O ENADE do ano 2000 possui quatro (4) quest\~oes centradas nos t\'opicos mais comumente estudados: integrais complexas, a equa\c c\~ao de Laplace, converg\^encia de s\'eries e matrizes. Muito centralizado no quesito mec\^anico na solu\c c\~ao dos problemas. O do ano 2001 tr\'as cinco (5) quest\~oes, a maioria na \'area de \'algebra, com t\'abuas de elementos de corpos, pergunta sobre algumas defini\c c\~oes de espa\c cos m\'etricos e sobre fun\c c\~oes, possui uma quest\~ao de aplica\c c\~ao de c\'alculo diferencial -na \'area de taxas de varia\c c\~ao e volume- e uma quest\~ao sobre a exponencial complexa. No ENADE do ano de 2002 encontram-se seis (6) quest\~oes discursivas, abrangendo a maioria dos t\'opicos principais da Matem\'atica, tais como, c\'alculo diferencial e integral, pedindo opera\c c\~oes com derivadas, s\'erie de pot\^encias -com o teste da raz\~ao- e a solu\c c\~ao de uma integral complexa, \'algebra, aritm\'etica, e geometria anal\'\i tica e vetores. O ENADE do ano de 2003 possui tamb\'em seis (6) quest\~oes, onde se deve escolher cinco (5) e resolv\^e-las, mas aqui estar\~ao todas resolvidas, a primeira quest\~ao \'e de c\'alculo, onde se pede a resolu\c c\~ao de uma integral dupla, as segunda, terceira e quarta quest\~oes, na \'area de \'algebra, pedem constru\c c\~oes de an\'eis, opera\c c\~oes com matrizes e polinômios, a quinta quest\~ao novamente sobre c\'alculo, desta vez, aborda campos vetoriais e necessita de manipula\c c\~oes \`a respeito dos divergente, convergente e laplaciano, encerrando com conjecturas sobre sequ\^encias na sexta quest\~ao. No prov\~ao do ENADE 2005 a \'unica quest\~ao aborda o tema de C\'alculo, onde se deve verificar as condi\c c\~oes de {\it Cauchy-Riemann} e realizar a solu\c c\~ao de uma integral complexa. O ENADE 2008 apresenta apenas uma (1) quest\~ao sobre o tema de An\'alise Matem\'atica, no que se diz respeito \`a diferencia\c c\~ao e propriedades de certas fam\'\i lias de fun\c c\~oes, tais como injetora e limitada. O ENADE 2011 cont\'em tr\^es (3) quest\~oes: a primeira quest\~ao aborda o tema de Estat\'\i stica, i.e. no c\'alculo de probabilidades, a segunda sequ\^encia, utilizando indu\c c\~ao finita para demonstrar uma conjectura e a terceira quest\~ao abrange a \'area de An\'alise Matem\'atica, onde primeiramente apresenta o {\it teorema do valor intermedi\'ario} e {\it a posteriori} o aplica nesse campo para resolver uma situa\c c\~ao-problema na Corrida de S\~ao Silvestre de 2010. As tr\^es (3) quest\~oes do ENADE de 2014 abordam os temas de geometria anal\'\i tica, matem\'atica aplicada e equa\c c\~oes Diofantinas, a primeira quest\~ao \'e sobre os efeitos visuais de uma transforma\c c\~ao da computa\c c\~ao gr\'afica, a segunda quest\~ao aborda o sistema de corre\c c\~ao de palavras de editores de texto com \'algebra, e a terceira pede para aplicar equa\c c\~oes diofantinas a um problema do cotidiano.

Num panorama geral percebemos o alto n\'\i vel de matem\'atica, n\~ao s\'o na parte discursiva, as quest\~oes necessitam de muita an\'alise e conhecimento sobre os principais teoremas dos assuntos abordados, tais como o teorema de {\it Green}, $\displaystyle \int_{\partial R} M dx + N dy= \int\!\int_R \left( \frac{\partial N}{\partial x} - \frac{\partial M}{\partial y}\right) dx dy $ para o c\'alculo de \'areas em algumas quest\~oes e o teorema de de {\it Cauchy} para a resolu\c c\~ao das integrais complexas do tipo $\displaystyle \int_C \frac{f(z)}{z-z_0} dz$ -presentes em todos os prov\~oes-, o teorema de {\it Cayley-Hamilton} para os processos envolvendo diagonaliza\c c\~ao e autovalores e autovetores, entre outras compet\^encias. Em todos os anos as quest\~oes abrangeram a maior parte da ementa de um curso de bacharelado, cumprindo com o objetivo da prova.

\chapter{Coment\'arios Gerais}

\part{Provas}
 
\chapter{ENADE 1998}

\section{\color{blue} Quest\~oes}

\subsection{\color{blue} Quest\~ao 1}

Seja $R$ uma regi\~ao do plano que satisfaz as condi\c c\~oes do Teorema de Green.

\begin{enumerate}

\item[(a)] Mostre que a \'area de $R$ \'e dada por $\displaystyle \frac1{2} \int_{\partial R} x dy - y dx$

\item[(b)] Use o item (a) para calcular a \'area da elipse de equa\c c\~oes $\begin{cases} x=a\cos(\theta) \\ y=b\sin(\theta)\end{cases}$ onde $a > 0$ e $b > 0$ s\~ao fixos, e $0 \leq \theta \leq 2 \pi$ (valor: 20,0 pontos)

\end{enumerate}

\paragraph{Dados/Informa\c c\~oes adicionais:} Teorema de Green: Seja $R$ uma regi\~ao do plano com interior n\~ao vazio e cuja fronteira $\partial R$ \'e formada por um n\'umero finito de curvas fechadas, simples, disjuntas e de classe $C^1$ por partes. Sejam $L(x,y)$ e $M(x,y)$ fun\c c\~oes de classe $C^1$ em $R$. Ent\~ao $\displaystyle \int\!\int_R \left(\frac{\partial M}{\partial x}-\frac{\partial L}{\partial y}\right)dx dy=\int_{\partial R} L dx + M dy$

\subsection{\color{blue} Quest\~ao 2}

Resolva a equa\c c\~ao diferencial $y'''- 4y'' + 4y' = e^x$, onde $y'=\displaystyle\frac{dy}{dx}$; $y''=\displaystyle\frac{d^2 y}{dx^2}$; $y'''=\displaystyle\frac{d^3 y}{dx^3}$ (valor: 20,0 pontos)

\subsection{\color{blue} Quest\~ao 3}

Prove que se uma seqü\^encia de fun\c c\~oes $f_n: D \to \mathbb R, D \subset R$ converge uniformemente para $f: D \to \mathbb R$ e cada $f_n$ \'e cont\'\i nua no ponto $a \in D$, ent\~ao $f$ \'e cont\'\i nua no ponto $a$.

\paragraph{Dados/Informa\c c\~oes adicionais:} Uma seqü\^encia de fun\c c\~oes $f_n: D \to \mathbb R, D \subset R$ converge uniformemente para $f: D \to \mathbb R$ se para todo $\epsilon > 0$ dado existe $n_0 \in \mathbb N$ tal que $n > n_0 \Longrightarrow |f_n(x) - f(x)| < \epsilon$ para todo $x \in D$. (valor: 20,0 pontos)

\subsection{\color{blue} Quest\~ao 4}

Seja $\gamma: [0,2\pi] \to \mathbb C$ a curva $\gamma (\theta) = e^{i\theta}$. Calcule $\displaystyle \int_\gamma \frac1{z-z_0} dz$ nos seguintes casos:

\begin{enumerate}

\item[(a)] $z_0=\displaystyle \frac1{2} (1+i)$

\item[(b)] $z_0 = 2(1 + i)$. (valor: 20,0 pontos)

\end{enumerate}

\subsection{\color{blue} Quest\~ao 5}

Sejam $\alpha$ um n\'umero alg\'ebrico de grau $n$ e  $\beta = b_0 + b_1\alpha + ... + b_{n-1}\alpha^{n-1}$ um elemento n\~ao nulo no corpo $\mathbb Q(\alpha)$, i.e., os coeficientes $b_i$ s\~ao racionais, $0 \leq i \leq n-1$, e, pelo menos, um deles \'e diferente de zero.

\begin{enumerate}

\item[(a)] Prove que $\displaystyle\frac1{\beta}$ \'e um polinômio em $\alpha$.

\item[(a)] Racionalize a fra\c c\~ao $\displaystyle \frac1{2+\sqrt[3]{2}}$. (valor: 20,0 pontos)

\end{enumerate}

\section{\color{red} Solu\c c\~oes}

\subsection{\color{red} Quest\~ao 1}

\begin{enumerate}

\item[(a)] A integral dada no enunciado nos fornece $L(x,y)=-y$ e $M(x,y)=x$. Calculando $\displaystyle \frac{\partial M}{\partial x}-\frac{\partial L}{\partial y}$ obtemos: 2. Como foi dito que a fun\c c\~ao satisfaz as condi\c c\~oes do Teorema de Green, ent\~ao a integral $\displaystyle \frac1{2} \int_{\partial R} x dy - y dx=\displaystyle \frac1{2}\int\!\int_R 2 dx dy=\int\!\int_R dx dy$, que corresponde \`a \'area da regi\~ao $R$.

\item[(b)] Temos ent\~ao $\displaystyle \frac1{2} \int_{\partial R} x dy - y dx=\frac1{2} \int_{\partial R} a\cos(\theta) dy-b\sin(\theta) dx$. Mas $dy=b \cos(\theta)d\theta$ e $dx=-a \sin(\theta) d\theta$, ent\~ao a integral se torna: 

$$\frac1{2} \int_{\partial R} a\cos(\theta) b \cos(\theta)d\theta-b\sin(\theta) (-a \sin(\theta)) d\theta=$$ $$=\frac{ab}{2} \int_{\partial R}\cos^2(\theta)+\sin^2(\theta) d\theta=\frac{ab}{2}\int_0^{2\pi} d\theta=ab\pi$$

\end{enumerate}

\subsection{\color{red} Quest\~ao 2}

Fazendo a substitui\c c\~ao: $u(x)=y'(x)$ a equa\c c\~ao diferencial assume a forma $u''-4u'+4u=e^x$. A solu\c c\~ao da equa\c c\~ao caracter\'\i stica \'e: $\lambda=2$, portanto a solu\c c\~ao da equa\c c\~ao homog\^enea associada \'e $u(x)=c_1e^{2x}+c_2xe^{2x}$.

Pela equa\c c\~ao n\~ao homog\^enea, uma aparente solu\c c\~ao \'e $u(x)=e^x$. De fato: $e^x-4e^x+4e^x=e^x$, portanto pelo princ\'\i pio da sobreposi\c c\~ao uma solu\c c\~ao da equa\c c\~ao diferencial \'e $u(x)=c_0e^x+c_1e^{2x}+c_2xe^{2x}$. Mas $u=y'$, ent\~ao $$y(x)=\int u(x) dx=\int c_0e^x+c_1e^{2x}+c_2xe^{2x}dx$$

Portanto a solu\c c\~ao da eq. diferencial \'e $y(x)=C_0e^x+C_1e^{2x}+ C_2 xe^{2x}+C_3$.

\subsection{\color{red} Quest\~ao 3}

Como a sequ\^encia de fun\c c\~oes converge para $f$, ent\~ao dado $\epsilon>0$ existe $n_o \in \mathbb N$ tal que para $n>n_0$, $|f_n(x)-f(x)|<\epsilon$.  Mas cada $f_n$ \'e cont\'\i nua no ponto $a$, ou seja, para $\delta>0$, $|x-a|<\delta$ implica que $|f_n(x)-f_n(a)|<\epsilon$. Como $f_n(x)$ converge para $f(x)$ ent\~ao $|f(x)-f(a)|<\epsilon$, portanto $f$ \'e cont\'\i nua em $a$.

\subsection{\color{red} Quest\~ao 4}

A curva em quest\~ao \'e a circunfer\^encia de raio $1$, ent\~ao:

\begin{enumerate}

\item[(a)] Como $z_0=\frac1{2}(1+i)$ est\'a dentro da curva $\gamma$, pois $|z_0|=\frac{\sqrt 2}{2}<1$, podemos usar o teorema de Cauchy para as integrais complexas, assim: $$\displaystyle \int_\gamma \frac1{z-\frac1{2}(1+i)} dz=2i\pi$$

\item[(b)] Como $z_0=2(1+i)$ est\'a fora da curva $\gamma$, pois $|z_0|=2\sqrt 2 > 1$, o valor da integral \'e zero.

\end{enumerate}

\subsection{\color{red} Quest\~ao 5}

\chapter{ENADE 1999}

\section{\color{blue} Quest\~oes}

\subsection{\color{blue} Quest\~ao 1}

Um modelo cl\'assico para o crescimento de uma popula\c c\~ao de determinada esp\'ecie est\'a descrito a seguir. Indicando por
$y = y(t)$ o n\'umero de indiv\'\i duos desta esp\'ecie, o modelo admite que a taxa de crescimento relativo da popula\c c\~ao seja proporcional
\`a diferen\c ca $M - y(t)$, onde $M > 0$ \'e uma constante. Isto conduz \`a equa\c c\~ao diferencial $\displaystyle \frac{y'}{y}= k (M - y)$, onde $k > 0$ \'e uma constante que depende da esp\'ecie. Com base no exposto:

\begin{enumerate}

\item[(a)] resolva a equa\c c\~ao diferencial acima; (valor: 10,0 pontos)

\item[(b)] considere o modelo apresentado para o caso particular em que $M = 1000$, $k = 1$ e $y (0) = 250$ e explique qualitativamente como se d\'a o crescimento da popula\c c\~ao correspondente, indicando os valores de t para os quais $y(t)$ \'e crescente, e o valor limite de $y(t)$ quando $t \to \infty$. (valor: 10,0 pontos)

\end{enumerate}

\subsection{\color{blue} Quest\~ao 2}

Seja $\mathbb Z_3 = {\bar 0 , \bar 1 , -\bar 1}$ o corpo de inteiros m\'odulo $3$ e $\mathbb Z_3 [x]$ o anel de polinômios em $x$ com coeficientes em $\mathbb Z_3$.

\begin{enumerate}

\item[(a)] Mostre que $x^2 + x - 1$ \'e irredut\'\i vel em $\mathbb Z_3 [x]$. (valor: 10,0 pontos)

\item[(b)] Mostre que o anel quociente $\displaystyle {\mathbb Z_3 [x]}/{x^2 + x - 1}$ \'e um corpo e que tem $9$ elementos. (valor: 10,0 pontos)

\end{enumerate}

\subsection{\color{blue} Quest\~ao 3}

Considere o subconjunto $\Gamma$ do $\mathbb R^2$ dado pela equa\c c\~ao $2(x^2+ y^2)^2=25(x^2-y^2)$.

\begin{enumerate}

\item[(a)] Para que valores de $x$ existem $v_x$ , vizinhan\c ca de $x$, e fun\c c\~ao diferenci\'avel $y = y(x)$ definida em $v_x$, satisfazendo $2(x^2+ y(x)^2)^2=25 (x^2-y(x)^2)$? Justifique. (valor: 10,0 pontos) 

\item[(b)] Obtenha a reta tangente a $\Gamma$ no ponto $(3, 1)$.
(valor: 10,0 pontos)

\end{enumerate}

\subsection{\color{blue} Quest\~ao 4}

Prove que se uma fun\c c\~ao $f: \mathbb R^n \to \mathbb R^n$ \'e cont\'\i nua, ent\~ao a imagem inversa $f^{-1}(V)$ de todo subconjunto aberto $V \subset \mathbb R^n$  \'e um subconjunto aberto de $\mathbb R^n$. (valor: 20,0 pontos)

\paragraph{Defini\c c\~ao:} Uma fun\c c\~ao $f: \mathbb R^n \to \mathbb R^n$ \'e cont\'\i nua num ponto $a \in \mathbb R^n$ quando, para todo $\epsilon > 0$ existe $d > 0$ tal que $|x - a| < \delta \Longrightarrow |f(x) - f(a)| < \epsilon$.

\subsection{\color{blue} Quest\~ao 5}

Sejam $\vec F: D \subset \mathbb R^n \to \mathbb R^n$ um campo conservativo, $\phi: D \subset \mathbb R^n \to \mathbb R^n$ uma fun\c c\~ao potencial de $\vec F$ e $\gamma:[a,b] \to D$ uma curva regular de classe $C^1$.

\begin{enumerate}

\item[(a)] Mostre que o trabalho realizado por $\vec F$ sobre $\gamma$ \'e dado por $\phi(\gamma(b)) - \phi(\gamma(a))$. (valor: 10,0 pontos)

\item[(b)] Calcule o trabalho realizado pelo campo $\vec F (x,y)=\displaystyle \left( \frac{x}{x^2+y^2},\frac{y}{x^2+y^2}  \right)$ sobre a curva esbo\c cada abaixo. (valor: 10,0 pontos)

\begin{center}
\begin{picture}(160,80)
\put(0,40){\vector(1,0){160}}
\put(150,30){$x$}
\put(60,0){\vector(0,1){80}}
\put(50,72){$y$}
\put(130,40){\circle*{3}}
\put(90,40){\circle*{3}}
\put(128,29){$e$}
\put(92,28){\small$1$}
\qbezier(130,40)(119,65)(60,66)
\qbezier(60,66)(28,65.5)(29,40)
\qbezier(29,40)(31,17)(60,16)
\qbezier(60,16)(86,17)(90,40)
\put(103,60.4){\vector(-4,1){2}}
\put(87.6,31.7){\vector(1,3){2}}
\end{picture}
\end{center}

\end{enumerate}

\paragraph{Defini\c c\~oes:} Um campo vetorial $\vec F: D \subset \mathbb R^n \to \mathbb R^n$ diz-se conservativo (ou gradiente) se existe $\phi: D \to \mathbb R$, de classe $C^1$, tal que $\vec \nabla \phi = \vec F$ em todo ponto de $D$. Uma tal $\phi$ chama-se fun\c c\~ao potencial. O trabalho realizado por um campo de vetores sobre uma curva $\gamma:[a,b] \to D$ \'e dado por $\displaystyle \int_a^b \vec F(\gamma(t))\cdot \vec \gamma '(t)dt$.

\section{\color{red} Solu\c c\~oes}

\subsection{\color{red} Quest\~ao 1}

\begin{enumerate}

\item[(a)] Dividindo ambos lados por $(M-y)$ e integrando em rela\c c\~ao a $t$ temos: $\displaystyle \int \frac{dy}{y(M-y)}=\int k dt$. A integral da direita \'e simplesmente $kt+c$. Mas na da esquerda precisamos fazer decomposi\c c\~ao em fra\c c\~oes parciais. Ent\~ao:

$${1\over y(M-y)}={A\over y}+{B\over M-y}={(B-A)y+AM\over y(M-y)}\Longrightarrow
\cases{A&=$1\over M$; \cr
B-A&$=0$}$$

Portanto nossa solu\c c\~ao para a decomposi\c c\~ao \'e: $A=\frac1{M}=B$. Ent\~ao nossa integral \'e: $$\frac1{M}\int \frac1
{y}+\frac1{M-y} dy=\frac1{M} \ln\left(\frac{y}{y-M}\right)$$

Assim nos reduzimos a: $\frac1{M} \ln\left(\frac{y}{y-M}\right)=kt+c \; \Longrightarrow \;\frac{y}{y-M}=e^{Mkt+Mc}$, ou seja: $y=(y-M)(e^c e^{kt})^M=y(e^c e^{kt})^M-M(e^c e^{kt})^M$, ent\~ao: $y((e^c e^{kt})^M-1)=M(e^c e^{kt})^M$, dividindo por $(e^c e^{kt})^M-1$ e chamando $e^{cM}=C$ e $kM=K$, finalmente temos: $$y=\frac{MC e^{Kt}}{C e^{Kt}-1}$$ multiplicando esta \'ultima equa\c c\~ao por $e^{-Kt}$ para cancelarmos duas exponenciais, a equa\c c\~ao assume a forma: $$y=\frac{MC}{C-e^{-Kt}}$$

\item[(b)] Sendo $M=1000$ e $k=1$, nossa constante \'e $K=1000$, e a equa\c c\~ao se torna $$y=\frac{1000C}{C-e^{-1000t}}$$ o enunciado nos deu $y(0)=250$, ent\~ao $250=\frac{1000C}{C-1}\Longrightarrow C=\frac{-5}{13}$. Ent\~ao nossa equa\c c\~ao se torna: $$y(t)=\frac{1000}{13e^{-1000t}/5+1}$$

A fun\c c\~ao \'e sempre crescente para valores positivos de $t$, e quando $t\to\infty$, $y\to 1000=M$.

\end{enumerate}

\subsection{\color{red} Quest\~ao 2}

\begin{enumerate}

\item[(a)] O polinômio $x^2+x-1$ \'e irredut\'\i vel pois $\bar 1^2+\bar 1 -1=\bar 1 \neq \bar 0$, $\bar 0^2+\bar 0-1=\bar 2\neq \bar 0$ e $\bar 2^2+\bar 2-1=\bar 2 \neq \bar 0$, logo n\~ao possui ra\'\i zes, ent\~ao \'e irredut\'\i vel.

\end{enumerate}

\subsection{\color{red} Quest\~ao 3}

\subsection{\color{red} Quest\~ao 4}

\subsection{\color{red} Quest\~ao 5}

\chapter{ENADE 2000}

\section{\color{blue} Quest\~oes}

\subsection{\color{blue} Quest\~ao 1}

Seja $\gamma$ um caminho no plano complexo, fechado, simples, suave (isto \'e, continuamente deriv\'avel) e que n\~ao passa por $i$ nem por $-i$. Quais s\~ao os poss\'\i veis valores da integral $\displaystyle \int_\gamma \frac{dz}{1+z^2}$? (valor: 20,0 pontos)

\subsection{\color{blue} Quest\~ao 2}

Uma fun\c c\~ao $u: \mathbb R^2 \to \mathbb R$, com derivadas cont\'\i nuas at\'e a 2\textordfeminine\ ordem, \'e dita harmônica em $\mathbb R^2$ se satisfaz a Equa\c c\~ao de Laplace: $$\Delta u={\partial^2u\over\partial x^2}+{\partial^2u\over\partial y^2}=0 \qquad \rm{em}\;\; \mathbb R^2$$ Mostre que se $u$ e $u^2$ s\~ao harmônicas em $\mathbb R^2$, ent\~ao $u$ \'e uma fun\c c\~ao constante. (valor: 20,0 pontos)

\subsection{\color{blue} Quest\~ao 3}

Seja $\{A_n\}, n\in \mathbb N$, uma seqü\^encia de n\'umeros reais positivos e considere a s\'erie de fun\c c\~oes de uma vari\'avel real $t$ dada por $\displaystyle \sum_{n=0}^\infty (A_n)^t$. Suponha que tal s\'erie converge se $t = t_0 \in \mathbb R$. Prove que ela converge uniformemente no intervalo $[t_0, \infty [$. (valor: 20,0 pontos)

\subsection{\color{blue} Quest\~ao 4}

Sejam $A =\left(\matrix{0 & -1 & 3 \cr 0 & 2 & 0 \cr 0 & -1 & 0}\right)$ e $n$ um inteiro positivo. Calcule $A^n$.

\paragraph{Sugest\~ao:} Use a Forma Canônica de Jordan ou o Teorema de Cayley-Hamilton. (valor: 20,0 pontos)

\section{\color{red} Solu\c c\~oes}

\subsection{\color{red} Quest\~ao 1}

\subsection{\color{red} Quest\~ao 2}

Como $u$ e $u^2$ s\~ao fun\c c\~oes harmônicas, ent\~ao: $$\frac{\partial^2u}{\partial x^2}+\frac{\partial^2u}{\partial y^2}=0$$ e $$\frac{\partial^2}{\partial x^2}(u^2)+\frac{\partial^2}{\partial y^2}(u^2)=0$$ como $\frac{\partial}{\partial x}(u^2)=2u\frac{\partial u}{\partial x}$, derivando novamente: $\frac{\partial}{\partial x}\left(2u\frac{\partial u}{\partial x}\right)=2\left(\frac{\partial u}{\partial x}\right)^2+2u\frac{\partial^2 u}{\partial x^2}$ e o mesmo acontece com a vari\'avel $y$, desse modo nossa equa\c c\~ao de Laplace toma a forma: $$\left(\frac{\partial u}{\partial x}\right)^2+u{\partial^2 u\over\partial x^2}+\left(\frac{\partial u}{\partial y}\right)^2+u{\partial^2 u\over\partial y^2}=\left({\partial u\over\partial x}\right)^2+\left({\partial u\over\partial y}\right)^2=0$$ pois  $u$ \'e harm\^onica. Portanto $\displaystyle{\partial u\over\partial x}=0$ e $\displaystyle\frac{\partial u}{\partial y}=0$. Resolvendo estas duas equa\c c\~oes, temos: (i) $u(x,y)= c+\phi(y)$ e (ii) $u(x,y)=k+\psi(x)$, derivando a primeira em rela\c c\~ao a $y$ e a segunda em rela\c c\~ao a $x$, obtemos: $\phi'(y)=0$ e $\psi'(x)=0$ respectivamente, o que indica que estas fun\c c\~oes s\~ao constantes. Assim necessariamente $\phi(y)=k$ e $\psi(x)=c$ e temos a unica solu\c c\~ao: $u(x,y)=c+k=C$

\subsection{\color{red} Quest\~ao 3}

\subsection{\color{red} Quest\~ao 4}

\chapter{ENADE 2001}

\section{\color{blue} Quest\~oes}

\subsection{\color{blue} Quest\~ao 1}

Sabendo-se que para todo n\'umero real $\theta$ tem-se que $e^{i\theta}= \cos (\theta) + i \sin (\theta)$, deduza as f\'ormulas

\begin{enumerate}

\item[(a)] $\sin (\alpha + \beta) = \sin (\alpha) \cos (\beta) + \cos (\alpha) \sin (\alpha)$ (valor: 10,0 pontos)

\item[(b)] $\cos (\alpha + \beta) = \cos (\alpha) \cos (\beta) - \sin (\alpha) \sin (\beta)$ (valor: 10,0 pontos)

\end{enumerate}

\subsection{\color{blue} Quest\~ao 2}

Uma piscina, vazia no instante $t = 0$, \'e abastecida por uma bomba d’\'agua cuja vaz\~ao no instante $t$ (horas) \'e $V(t)$ (metros c\'ubicos por hora).

\begin{enumerate}

\item[(a)] Determine o volume da piscina sabendo que, se $V(t) = 500$, a piscina fica cheia em $5$ horas. (valor: 5,0 pontos)

\item[(b)] Determine em quanto tempo a piscina ficaria cheia se $V(t) = 50 t$. (valor: 15,0 pontos)

\end{enumerate}

\subsection{\color{blue} Quest\~ao 3}

Sejam $A$ uma matriz real $2 \times 2$ com autovalores $\frac1{2}$ e $\frac1{3}$ e $\textbf v$ um vetor de $\mathbb R^2$.

\begin{enumerate}

\item[(a)] $A$ \'e diagonaliz\'avel? Justifique sua resposta. (valor: 5,0 pontos)

\item[(b)] Considere a seqü\^encia $\textbf v , A\textbf v , A^2\textbf v , A^3\textbf v , ... , A^n\textbf v , ... $. Prove que essa seqü\^encia \'e convergente. (valor: 15,0 pontos)

\end{enumerate}

\subsection{\color{blue} Quest\~ao 4}

Sejam $X$ e $Y$ espa\c cos m\'etricos, $A \subset X$ e $f: X \to Y$ uma fun\c c\~ao.

\begin{enumerate}

\item[(a)] Qual \'e o significado de “$A$ \'e aberto”? (valor: 5,0 pontos)

\item[(b)] Qual \'e o significado de “$A$ \'e fechado”? (valor: 5,0 pontos)

\item[(c)] Qual \'e o significado de “$f$ \'e cont\'\i nua em $X$”? (valor: 5,0 pontos)

\item[(d)] Se $a \in Y$ e $f$ \'e cont\'\i nua em $X$, mostre que o conjunto solu\c c\~ao da equa\c c\~ao $f(x) = a$ \'e fechado. (valor: 5,0 pontos)

\end{enumerate}

\subsection{\color{blue} Quest\~ao 5}

O corpo $\mathbb Z_2$ dos inteiros m\'odulo $2$ \'e formado por dois elementos, $0$ e $1$, com as opera\c c\~oes usuais de adi\c c\~ao e multiplica\c c\~ao definidas pelas t\'abuas abaixo.

$$\begin{array}{|c|c|c|}
\hline + & 0 & 1 \\
\hline 0 & 0 & 1 \\
\hline 1 & 1 & 0 \\
\hline
\end{array} \qquad \qquad
\begin{array}{|c|c|c|}
\hline \times & 0 & 1 \\
\hline 0 & 0 & 0 \\
\hline 1 & 0 & 1 \\
\hline
\end{array}$$

Considere em $\mathbb Z_2[x]$ – isto \'e, no anel dos polinômios na indeterminada $x$ cujos coeficientes pertencem a $\mathbb Z_2$ –, o polinômio de grau $2$, $q (x) = x^2 + x + 1$.

\begin{enumerate}

\item[(a)] Mostre que $q(x)$ n\~ao tem ra\'\i zes em $\mathbb Z_2$. (valor: 5,0 pontos)

\item[(b)] $q(x)$ sendo irredut\'\i vel, sabe-se, pelo Teorema de Kronecker, que existem um corpo $E$, que \'e uma extens\~ao de $\mathbb Z_2$ (ou seja, tal que $\mathbb Z_2$ \'e um subcorpo de $E$) e um elemento $\alpha \in E$ tal que $\alpha \notin \mathbb Z_2$ e $q(\alpha) = 0$. Determine o n\'umero m\'\i nimo de elementos que $E$ pode ter e construa as t\'abuas de adi\c c\~ao e de multiplica\c c\~ao em $E$. (valor: 15,0 pontos)

\end{enumerate}

\section{\color{red} Solu\c c\~oes}

\subsection{\color{red} Quest\~ao 1}

\begin{enumerate}

\item[(a)]

\end{enumerate}

\subsection{\color{red} Quest\~ao 2}

\begin{enumerate}

\item[(a)] $V=500\times 5=2500$ (metros c\'ubicos)

\item[(b)] $2500=\displaystyle \int_0^{t_1} 50 t dt \longrightarrow 2500=\frac{50t_1^2}{2}$, ou seja: $t_1=10$ (horas).

\end{enumerate}

\subsection{\color{red} Quest\~ao 3}

\begin{enumerate}

\item[(a)] $q(0)=1\neq 0$ e $q(1)=3=1\neq 0$ portanto \'e irredut\'\i vel.

\item[(b)] o Conjunto $E$ que se diz \'e $\mathbb Z_2$ extendido com as ra\'\i zes de $q(x)$, ou seja $\alpha=\displaystyle\frac{-1\pm \sqrt{1-4}}{2}$

\end{enumerate}

\subsection{\color{red} Quest\~ao 4}

\subsection{\color{red} Quest\~ao 5}

\chapter{ENADE 2002}

\section{\color{blue} Quest\~oes}

\subsection{\color{blue} Quest\~ao 1}

Sejam $g$ e $h$ fun\c c\~oes deriv\'aveis de $\mathbb R$ em $\mathbb R$ tais que $g’(x)=h(x)$, $h’(x)=g(x)$, $g(0)=0$ e $h(0)=1$.

\begin{enumerate}

\item[(a)] Calcule a derivada de $h^2(x) - g^2(x)$. (valor: 10,0 pontos) 

\item[(a)] Mostre que $h^2(x) - g^2(x) = 1$, para todo $x$ em $\mathbb R$. (valor: 10,0 pontos)

\end{enumerate}

\subsection{\color{blue} Quest\~ao 2}

Em um espa\c co m\'etrico $M$, com dist\^ancia $d$, a bola aberta de raio $r > 0$ e centro $p\in M$ \'e o conjunto $B_r(p) = \{x\in M | d(x,p) < r\}$. Por defini\c c\~ao, um conjunto $A \subset M$ \'e aberto se para qualquer ponto $p \in A$ existir $\epsilon > 0$ tal que $B_\epsilon(p) \subset A$.

\begin{enumerate}

\item[(a)] Mostre que a uni\~ao de uma fam\'\i lia qualquer de conjuntos abertos \'e um conjunto aberto. (valor: 5,0 pontos)

\item[(b)] Mostre que a interse\c c\~ao de uma fam\'\i lia finita n\~ao vazia de conjuntos abertos \'e um conjunto aberto. (valor: 10,0 pontos) 

\item[(c)] Em $\mathbb R$, com a m\'etrica usual, o conjunto $\{0\}$ n\~ao \'e aberto. D\^e exemplo de uma fam\'\i lia infinita de conjuntos abertos de R cuja interse\c c\~ao seja $\{0\}$. (valor: 5,0 pontos)

\end{enumerate}

\subsection{\color{blue} Quest\~ao 3}

Seja $A$ uma matriz quadrada de ordem $n$.

\begin{enumerate}

\item[(a)] Defina autovalor de $A$. (valor: 5,0 pontos)

\item[(b)] Se $\lambda$ \'e um autovalor de $A$, mostre que $2\lambda$ \'e um autovalor de $2A$. (valor: 5,0 pontos)

\item[(c)] Se $\lambda$ \'e um autovalor de $A$, mostre que $\lambda ^2$ \'e um autovalor de $A^2$. (valor: 10,0 pontos)

\end{enumerate}

\subsection{\color{blue} Quest\~ao 4}

O complexo $w$ \'e tal que a equa\c c\~ao $z^2-wz+(1-i)=0$ admite $1+i$ como raiz. 

\begin{enumerate}

\item[(a)] Determine $w$. (valor: 5,0 pontos)

\item[(b)] Determine a outra raiz da equa\c c\~ao. (valor: 5,0 pontos)

\item[(c)] Calcule a integral $\displaystyle \int_\gamma \frac{dz}{z^2-wz+(1-i)}$, sendo $\gamma$ a circunfer\^encia descrita parametricamente por $\gamma(t)=\frac1{2} \cos(t)+i(\frac1{2}\sin(t)-1)$, $0\leq t \leq 2\pi$. (valor: 10,0 pontos)

\end{enumerate}

\subsection{\color{blue} Quest\~ao 5}

A s\'erie de pot\^encias a seguir define, no seu intervalo de converg\^encia, uma fun\c c\~ao $g$, $g(x) =\displaystyle 1-\frac{x^2}{2}+\frac{x^4}{4}-...+(-1)^n\frac{x^{2n}}{2n}+...$

\begin{enumerate}

\item[(a)] Determine o raio de converg\^encia $r$ da s\'erie. Justifique. (valor: 5,0 pontos) 

\item[(b)] Expresse $g'(x)$ como soma de uma s\'erie de pot\^encias, para $|x|<r$. (valor: 5,0 pontos)

\item[(c)] Expresse $g'(x)$, para $|x|<r$, em termos de fun\c c\~oes elementares (polinomiais, trigonom\'etricas, logar\'\i tmicas, exponenciais). (valor: 5,0 pontos)

\item[(d)] Expresse $g(x)$, para $|x|<r$, em termos de fun\c c\~oes elementares. (valor: 5,0 pontos)

\end{enumerate}

\subsection{\color{blue} Quest\~ao 6}

Uma fonte de luz localizada no ponto $L = (0,-1, 0)$ ilumina a superf\'\i cie dada, parametricamente, por $P(u,v) = (u + v, u^2, v)$.

\begin{enumerate}

\item[(a)] Calcule o vetor normal \`a superf\'\i cie, $\vec N (u,v)$, de forma que para $u = v = 0$ esse vetor seja $(0,-1, 0)$. (valor: 5,0 pontos)

\item[(b)] Trabalhando com os vetores $\vec N$ e $L-P$, d\^e uma condi\c c\~ao sobre $u$ e $v$ a fim de que o ponto $P(u,v)$ seja iluminado pela luz em $L$. (valor: 15,0 pontos)

\end{enumerate}

\section{\color{red} Solu\c c\~oes}

\subsection{\color{red} Quest\~ao 1}

\begin{enumerate}

\item[(a)] $(h^2(x)-g^2(x))'=2hh'-2gg'=2hg-2gh=0$

\item[(b)] Como $(h^2(x)-g^2(x))'=0$, temos que $h^2(x)-g^2(x)=k$. Mas $h(0)=1$ e $g(0)=0$ ent\~ao: $1^2-0^2=1\Longrightarrow k=1$

\end{enumerate}

\subsection{\color{red} Quest\~ao 2}

\subsection{\color{red} Quest\~ao 3}

\subsection{\color{red} Quest\~ao 4}

\begin{enumerate}

\item[(a)] $(1+i)^2-w(1+i)+(1-i)=0\Longrightarrow w=1$

\item[(b)] Como o enunciado disse que $1+i$ \'e raiz, ent\~ao podemos rescrever a equa\c c\~ao dada. Assim: $z^2-z+(1-i)=0$, por inspe\c c\~ao percebemos que a outra raiz \'e $-i$, pois $(-i)^2-(-i)+(1-i)=-1+i+(1-i)=0$.

\item[(c)] $\displaystyle \int_\gamma \frac{dz}{z^2-wz+(1-i)}= \int_\gamma \frac{dz}{(z-1+i)(z+i)}$, decompondo com fra\c c\~oes parciais temos: $$\frac1{(z-1+i)(z+i)}=\frac{a+bi}{(z-1+i)}+\frac{c+di}{(z+i)}$$

Onde obtemos: $\displaystyle \frac{(a+bi+c+di)z+(-b-c-d)+(c-b+a-d)i}{(z-1+i)(z+i)}$.

A qual gera o seguinte sistema: $$\cases{a+bi+c+di&$=0$ \cr b+c+d&$=-1$ \cr c-b+a-d&$=0$\cr}$$

A primeira equa\c c\~ao nos d\'a que $a=-c$ e $b=-d$, substituindo na segunda equa\c c\~ao temos: $c=-1$, logo, $a=1$. Portanto as constantes s\~ao: $a=1$, $c=-1$, nosso sistema agora \'e apenas a equa\c c\~ao $b+d=0$ Portanto podemos fazer $b=d=0$. Assim as fra\c c\~oes s\~ao $\displaystyle \frac1{z-1+i}-\frac1{z+i}$, e a integral original torna-se $\displaystyle \int_\gamma \frac1{z-(1-i)}-\frac1{z+i}dz$. A circunfer\^encia $\gamma$ pode ser rescrita como: $\gamma(t)=\frac1{2}[\cos(t)+i\sin(t)]-i$, desse modo percebemos que est\'a centrada no ponto $(0,-i)$ e possui raio $1/2$. Como a primeira raiz est\'a fora da curva, o valor da integral \'e zero. Ent\~ao: $$\int_\gamma \frac1{z-(1-i)}-\frac1{z+i}dz=-\int_\gamma\frac1{z+i}dz=-2i\pi$$

\end{enumerate}

\subsection{\color{red} Quest\~ao 5}

\begin{enumerate}

\item[(a)] Pelo teste da raz\~ao temos: $$\lim_{n\to\infty}\left|\frac{(-1)^{n+1}\displaystyle \frac{x^{2(n+1)}}{2(n+1)}}{(-1)^n\displaystyle \frac{x^{2n}}{2n}}\right|=\lim_{n\to\infty}\left|\frac{x^{2(n+1)}}{2(n+1)}\frac{2n}{x^{2n}}\right|.$$

Assim $\lim_{n\to\infty} \left|\frac{n}{n+1}x^2\right|$ onde percebemos que $|x^2|<1$, ou simplesmente $|x|<1$. Ent\~ao o raio de convergencia $r$ da s\'erie \'e: $r<1$.

\item[(b)] Derivando: $g'(x)=\displaystyle -x+x^3-x^5+...+(-1)^n x^{2n-1}+...$

\item[(c)] Considere a s\'erie de {\it Maclaurin} da fun\c c\~ao $y=\ln(x+1)$, i. e. : $$\ln(x+1)=x - \frac1{2} x^2 + \frac1{3}x^3 - \frac1{4}x^4 + \frac1{5}x^5-\ldots$$ se utilizarmos $x^2$ no lugar de $x$, ter\'\i amos:

\begin{eqnarray*}
{d\over dx}\ln(x^2+1)&=&{2x\over x^2+1}\\
{d^2\over dx^2}\ln(x^2+1)&=&{2(x^2+1)-4x^2\over(x^2+1)^2}\\
&=&{1-2x^2\over(x^2+1)^2}
\end{eqnarray*}

\end{enumerate}

\subsection{\color{red} Quest\~ao 6}

\chapter{ENADE 2003}

\section{\color{blue} Quest\~oes}

\subsection{\color{blue} Quest\~ao 1}

Seja $I =\displaystyle \int_0 ^3\int_{\sqrt{\frac{x}{3}}}^1 e^{y^3} dydx$.

\begin{enumerate}

\item[(a)] Esboce graficamente a regi\~ao de integra\c c\~ao. (valor: 5,0 pontos)

\item[(b)] Inverta a ordem de integra\c c\~ao. (valor: 10,0 pontos)

\item[(c)] Calcule o valor de $I$. (valor: 5,0 pontos)

\end{enumerate}

\subsection{\color{blue} Quest\~ao 2}

Seja $\mathbb{Z}_{18}$ o anel dos inteiros m\'odulo 18 e seja $G$ o grupo multiplicativo dos elementos invert\'\i veis de $\mathbb{Z}_{18}$.

\begin{enumerate}

\item[(a)] Escreva todos os elementos do grupo $G$. (valor: 10,0 pontos)

\item[(b)] Mostre que $G$ \'e c\'\i clico, calculando explicitamente um gerador, ou seja, mostre que existe $g \in G$ tal que todos os elementos de $G$ s\~ao pot\^encias de $g$. (valor: 10,0 pontos)

\end{enumerate}

\subsection{\color{blue} Quest\~ao 3}

\begin{enumerate}

\item[(a)] Dada a matriz sim\'etrica $A = \matrix{ 1 & 6 \cr 6 & 4}$, escreva, em forma de polinômio $f(x,y)$, a forma quadr\'atica definida por $A$, isto \'e, calcule os coeficientes num\'ericos de
$f(x,y) = v^t A v$ , onde $v = \matrix{ x \cr y}$ e $v^t$ significa “$v$ transposto”. (valor: 5,0 pontos)

\item[(b)] Encontre uma matriz invert\'\i vel $P$ tal que $P^t A P = D$, onde $D$ \'e uma matriz diagonal. Para isto, basta tomar como $P$ uma matriz que tenha por colunas um par de autovetores ortonormais de $A$. (valor: 10,0 pontos)

\item[(c)] Na forma quadr\'atica $f(x,y) = v^t A v$, fa\c ca uma transforma\c c\~ao de coordenadas $v = P \tilde v$, sendo $\tilde v = \matrix{\tilde x \cr \tilde y}$, obtendo a forma quadr\'atica diagonalizada, isto \'e, sem o termo em $\tilde x \tilde y$ . (valor: 5,0 pontos)

\end{enumerate}

\subsection{\color{blue} Quest\~ao 4}

Seja $p(x) = x^n + a_{n-1} x^{n-1} + ... + a_1 x + a_0$ , com $n \geq 1$, um polinômio de coeficientes reais. Suponha que $p'(x)$ divide $p(x)$.

\begin{enumerate}

\item[(a)] Prove que o quociente $q(x) = \displaystyle \frac{p(x)}{p'(x)}$ \'e da forma $q(x) = \frac1{n} (x-x_0), x_0 \in \mathbb R$. (valor: 5,0 pontos)

\item[(b)] Encontre todos os polinômios $p(x)$ que satisfazem essa condi\c c\~ao, resolvendo a equa\c c\~ao diferencial $q(x) p'(x)-p(x) = 0$. (valor: 15,0 pontos)

\end{enumerate}

\subsection{\color{blue} Quest\~ao 5}

Dado um conjunto aberto $U \subset \mathbb R^3$ e um campo de vetores $X = (X_1 , X_2 , X_3 ): U \to \mathbb R^3$ diferenci\'avel, o divergente de $X$ \'e definido por $$div X=\frac{\partial X_1}{\partial x}+\frac{\partial X_2}{\partial y}+\frac{\partial X_3}{\partial z}$$

Para uma fun\c c\~ao de classe $C^2, f: U \to \mathbb R^3$ o laplaciano de $f$ \'e definido por $$\Delta f = \frac{\partial^2 f}{\partial x^2}+\frac{\partial^2 f}{\partial y^2}+\frac{\partial^2 f}{\partial z^2}$$

\begin{enumerate}

\item[(a)] Se $f: U \to \mathbb R$ \'e diferenci\'avel e $X: U \to \mathbb R^3$ \'e um campo de vetores diferenci\'avel, mostre que $$div(f X)=f\; div(X) + \nabla f \cdot X ,$$ sendo $\nabla f$ o gradiente de $f$ e $\nabla f \cdot X$ o produto interno entre $\nabla f$ e $X$. (valor: 5,0 pontos)

\item[(b)] Se $f: U \to \mathbb R$ \'e de classe $C^2$, mostre que $div(f\nabla f) = f \Delta f + || \nabla f ||^2$, sendo $||\;\;||$ a norma euclidiana. (valor: 5,0 pontos)

\item[(c)] Se $U = B = \{x \in \mathbb R^3: || x || < 1\}$ e $f: \bar B \to \mathbb R$ \'e de classe $C^3$ tal que $f(x) > 0$ para qualquer $x \neq 0$, $div(f \nabla f) = 5f$ e $|| \nabla f ||^2= 2f$, calcule $$ \int_S \frac{\partial f}{\partial N} dS ,$$ onde $\bar B$ \'e o fecho de $B$, $S$ \'e a fronteira de $B$, $N$ \'e a norma unit\'aria exterior a $S$, $\displaystyle \frac{\partial f}{\partial N}$ \'e a derivada direcional de $f$ na dire\c c\~ao de $N$ e $dS$ \'e o elemento de \'area de $S$. (valor: 10,0 pontos)

\end{enumerate}

\subsection{\color{blue} Quest\~ao 6}

Considere a fun\c c\~ao real $f$ definida, para $x \geq 0$, por $f(x) = \sqrt{2x}$.

\begin{enumerate}

\item[(a)] Prove que se $0 < x < 2$, ent\~ao $x < f(x) < 2$. (valor: 5,0 pontos)

\item[(b)] Prove que \'e convergente a seqü\^encia definida recursivamente por

\begin{enumerate}

\item[1.] $a_1=\sqrt 2$

\item[2.] $a_{n+1}=f(a_n)$, para todo $n \geq 1$

\end{enumerate}

(valor: 5,0 pontos)

\item[(c)] Calcule $\displaystyle \lim_{n \to \infty} a_n$ (valor: 10,0 pontos)

\end{enumerate}

\section{\color{red} Solu\c c\~oes}

\subsection{\color{red} Quest\~ao 1}

\begin{enumerate}

\item[(a)]

\begin{center}
\begin{picture}(200,150)
\put(0,20){\vector(1,0){200}}
\put(195,10){$x$}
\put(20,0){\vector(0,1){150}}
\put(9,145){$y$}
\put(100,52){\small $y=\sqrt{\frac{x}{3}}$}
\multiput(186,0)(0,11){14}{\line(0,1){5}}
\put(186,80){\circle*{3}}
\put(10,77){\small{$1$}}
\put(176,8){\small{$3$}}
\color{blue}
\qbezier(20,20)(20,50)(200,83)
\put(20,80){\line(1,0){164}}
\put(20,20){\line(0,1){60}}
\put(50,60){$R$}
\end{picture}
\end{center}

\item[(b)] $I =\displaystyle \int_0 ^3\int_{\sqrt{\frac{x}{3}}}^1 e^{y^3} dydx=\displaystyle \int_0 ^1\int_0^{3y^2} e^{y^3} dxdy$

\item[(c)] $I=\displaystyle \int_0 ^1\int_0^{3y^2} e^{y^3} dxdy=\int_0^13y^2e^{y^3}dy=e-1$ 

\end{enumerate}

\subsection{\color{red} Quest\~ao 2}

\begin{enumerate}

\item[(a)] $G=\{ 1,5,7,11,13 \}$

\end{enumerate}

\subsection{\color{red} Quest\~ao 3}



\subsection{\color{red} Quest\~ao 4}

\begin{enumerate}

\item[(a)] Temos que $p'(x)=nx^{n-1}+a_{n-1}(n-1)x^{n-2}+...+a_1$. Como $p'(x)$ divide $p(x)$ ent\~ao $q(x)$ deve ser de grau um. Portando $q(x)=k(x-x_0)$.

O teorema fundamental da divis\~ao nos d\'a: $q(x)p'(x)=p(x)$, assim $k(x-x_0)[nx^{n-1}+a_{n-1}(n-1)x^{n-2}+...+a_1]=x^n + a_{n-1} x^{n-1} + ... + a_1 x + a_0$. Multiplicando o primeiro termo da esquerda temos $knx^n=x^n$, portanto $k=\frac1{n}$. Logo $q(x)= \frac1{n}(x-x_0)$.

\item[(b)] Multiplicando a equa\c c\~ao diferencial dada no enunciado por $\frac1{q(x)}$ obtemos: $$p'(x)-\frac{p(x)}{q(x)}=0$$ ent\~ao $\frac{p'}{p}=\frac{n}{x-x_0} \Longrightarrow \ln(p)=n\ln(x-x_0)+c$, portanto $p(x)=k(x-x_0)^n$.

\end{enumerate}

\subsection{\color{red} Quest\~ao 5}



\subsection{\color{red} Quest\~ao 6}

\chapter{ENADE 2005}

\section{\color{blue} Quest\~oes}

\subsection{\color{blue} Quest\~ao 1}

A respeito de fun\c c\~oes de vari\'avel complexa, resolva os itens que se seguem.

\begin{enumerate}

\item[(a)] Escreva a fun\c c\~ao complexa $f(z) = f(x + iy) = z^2 -3z + 5$ na forma $f(z) = u(x, y) + i v(x, y)$ e verifique as equa\c c\~oes de Cauchy-Riemann para essa fun\c c\~ao. (valor: 4,0 pontos)

\item[(b)] Sabendo que $g(z)=\displaystyle \frac1{(z^2+1)(z+1)^2}=-\frac1{4(z-i)}-\frac1{4(z+i)}+\frac1{2(z+1)^2}+\frac1{2(z+1)}$, calcule a integral complexa: $\displaystyle \int_{|z|=2}\frac{z^2}{(z^2+1)(z+1)^2}dz$. (valor: 6,0 pontos)

\end{enumerate}

\section{\color{red} Solu\c c\~oes}

\subsection{\color{red} Quest\~ao 1}

\begin{enumerate}

\item[(a)] Fazendo $z=x+yi$ temos: $f(z)=(x+yi)^2-3(x+yi)+5$, ou seja: $f(z)=(5-3x+x^2-y^2)+(2xy-3y)i$. Daqui temos que $u(x,y)=5-3x+x^2-y^2$ e $v(x,y)=2xy-3y$; As condi\c c\~oes de Cauchy-Riemann s\~ao: $\displaystyle \frac{\partial u}{\partial x}=\frac{\partial v}{\partial y}$ e $\displaystyle \frac{\partial v}{\partial x}=-\frac{\partial u}{\partial y}$. Portanto teremos: $\displaystyle \frac{\partial u}{\partial x}=2x-3$ e $\displaystyle \frac{\partial v}{\partial y}=2x-3$, onde vemos $\displaystyle \frac{\partial u}{\partial x}=\frac{\partial v}{\partial y}$, e $\displaystyle \frac{\partial v}{\partial x}=2y$, $\displaystyle-\frac{\partial u}{\partial y}=2y$, portanto a fun\c c\~ao satisfaz as condi\c c\~oes citadas.

\item[(b)] Usando a sugest\~ao dada no enunciado vemos que as singularidades $-1, -i, i$ est\~ao contidas na curva fechada $C:|z|=2$, assim podemos usar o teorema de Cauchy, i. e. $\displaystyle \int_C \frac{f(z)}{(z-a)^{n+1}} dz=\frac{2i\pi}{ n!}f^{(n)}(a)$. Multiplicando a fun\c c\~ao $g(z)$ por $z^2$ temos a fun\c c\~ao desejada na integra\c c\~ao, assim $f(z)=z^2$ e: 

$$\int_{|z|=2}\frac{z^2}{(z^2+1)(z+1)^2}dz$$

$$=2i\pi\left(-\frac{(i)^2}{4}-\frac{(-i)^2}{4}+\frac{2(-1)}{2}+\frac{(-1)^2}{2}\right)$$

$$=\frac{i\pi}{2}+\frac{i\pi}{2}-2i\pi+i\pi=0$$

\end{enumerate}

\chapter{ENADE 2008}

\section{\color{blue} Quest\~oes}

\subsection{\color{blue} Quest\~ao 1}

Considere uma fun\c c\~ao deriv\'avel $f: \mathbb{R} \to \mathbb{R}$ que satisfaz \`a seguinte condi\c c\~ao:

Para qualquer n\'umero real $k\neq 0$, a fun\c c\~ao $g_k (x)$ definida por $g_k (x)=x-kf(x)$ n\~ao \'e injetora.

Com base nessa propriedade, fa\c ca o que se pede nos itens a seguir e transcreva suas respostas para o Caderno de Respostas, nos locais devidamente indicados.

\begin{enumerate}

\item[(a)] Mostre que, se $g'_k(x_0)=0$ para algum $k\neq 0$, ent\~ao $f' (x_0)=\frac1{k}$ (valor: 3,0 pontos).

\item[(b)] Mostre que, para cada $k \in \mathbb{R}$ n\~ao-nulo, existem n\'umeros $\alpha_k$ e $\beta_k$ tais que $g_k(\alpha_k) = g_k(\beta_k)$. Al\'em disso, justifique que, para todo $k \in \mathbb{R}$ n\~ao-nulo, existe um n\'umero $\theta_k$ tal que $g'_k(\theta_k)=0$. (valor: 3,0 pontos).

\item[(c)] Mostre que a fun\c c\~ao derivada de primeira ordem $f'$ n\~ao \'e limitada. (valor: 4,0 pontos).

\end{enumerate}

\section{\color{red} Solu\c c\~oes}

\subsection{\color{red} Quest\~ao 1}

\begin{enumerate}

\item[(a)] Derivando a fun\c c\~ao definida no item (a): $g'_k(x)=1-kf'(x)$. Fazendo $g'_k(x)=0$ temos: $0=1-kf'(x)$, ou seja: $f'(x_0)=\frac1{k}$, para um certo $x_0$

\item[(b)] Como o exerc\'\i cio nos diz que a fun\c c\~ao $g_k(x)$ n\~ao \'e injetora, essa defini\c c\~ao implica que existem $\alpha$ e $\beta$, diferentes, tais que: $g_k(\alpha)=g_k(\beta)$, mas como a mudan\c ca do valor de $k$ gera novas fun\c c\~oes injetoras, \'e cômodo escrever $g_k(\alpha_k)=g_k(\beta_k)$ para mostrar tal fato; Usando o resultado do item (a), temos que se $g'_k(\theta_k)=0$ ent\~ao $f'(\theta_k)=\frac1{k}$, portanto para cada valor de $k\neq 0$ temos uma fun\c c\~ao $g'_k(\theta_k)=0$

\item[(c)] A fun\c c\~ao $f'$ n\~ao \'e limitada pois a fun\c c\~ao $1/x$ , para $x \neq 0$ n\~ao \'e limitada.

\end{enumerate}

\chapter{ENADE 2011}

\section{\color{blue} Quest\~oes}

\subsection{\color{blue} Quest\~ao 1}

Em um pr\'edio de 8 andares, 5 pessoas aguardam o elevador no andar t\'erreo. Considere que elas entrar\~ao no elevador e sair\~ao, de maneira aleat\'oria, nos andares de 1 a 8.

Com base nessa situa\c c\~ao, fa\c ca o que se pede nos itens a seguir, apresentando o procedimento de c\'alculo utilizado na sua resolu\c c\~ao.

\begin{enumerate}

\item[(a)] Calcule a probabilidade de essas pessoas descerem em andares diferentes. (valor: 6,0 pontos).

\item[(b)] Calcule a probabilidade de duas ou mais pessoas descerem em um mesmo andar. (valor: 4,0 pontos).

\end{enumerate}

\subsection{\color{blue} Quest\~ao 2}

Considere a sequ\^encia num\'erica definida por $$\cases{a_1 = a;&\cr
a_{n+1}= \displaystyle{4a_n\over2+a_n^2}, & para $n\geq 1.$\cr}$$ 

Use o princ\'\i pio de indu\c c\~ao finita e mostre que $a_n<\sqrt{2}$, para todo n\'umero natural $n\geq 1$ e para $0<a<\sqrt{2}$, seguindo os passos indicados nos itens a seguir:

\begin{enumerate}

\item[(a)] escreva a hip\'otese e a tese da propriedade a ser demonstrada; (valor: 1,0 ponto)

\item[(b)] mostre que $\displaystyle{s = \frac{4a}{2+a^2}>0}$, para todo $a>0$; (valor: 1,0 ponto)

\item[(c)] prove que $s^2<2$, para todo $0<a<\sqrt{2}$; (valor: 3,0 pontos)

\item[(d)] mostre que $0<s<\sqrt{2}$; (valor: 2,0 pontos)

\item[(e)] suponha que $a_n<\sqrt{2}$ e prove que $a_{n+1}<2$; (valor: 1,0 ponto)

\item[(f)] conclua a prova por indu\c c\~ao. (valor: 2,0 pontos)

\end{enumerate}

\subsection{\color{blue} Quest\~ao 3}

O Teorema do Valor Intermedi\'ario \'e uma proposi\c c\~ao muito importante da an\'alise matem\'atica, com in\'umeras aplica\c c\~oes te\'oricas e pr\'aticas. Uma demonstra\c c\~ao anal\'\i tica desse teorema foi feita pelo matem\'atico Bernard Bolzano [1781 – 1848]. Nesse contexto, fa\c ca o que se pede nos itens a seguir:

\begin{enumerate}

\item[(a)] Enuncie o Teorema do Valor Intermedi\'ario para fun\c c\~oes reais de uma vari\'avel real; (valor: 2,0 pontos)

\item[(b)] Resolva a seguinte situa\c c\~ao-problema.

O vencedor da corrida de S\~ao Silvestre-2010 foi o brasileiro Mailson Gomes dos Santos, que fez o percurso de 15 km em 44 min e 7 seg. Prove que, em pelo menos dois momentos distintos da corrida, a velocidade instant\^anea de Mailson era de 5 metros por segundo. (valor: 4,0 pontos)

\item[(c)] Descreva uma situa\c c\~ao real que pode ser modelada por meio de uma fun\c c\~ao cont\'\i nua $f$, definida em um intervalo $[a , b]$, relacionando duas grandezas $x$ e $y$, tal que existe $k\in (a , b)$ com $f(x) \neq f(k)$, para todo $x\in (a , b), x \neq k$. Justifique sua resposta. (valor: 4,0 pontos)

\end{enumerate}

\section{\color{red} Solu\c c\~oes}

\subsection{\color{red} Quest\~ao 1}

\begin{enumerate}

\item[(a)] Considerando que as pessoas escolhem de forma aleat\'oria o andar que desejam ir, cada uma das pessoas t\^em 8 possibilidades, totalizando, pelo {\it princ\'\i pio multiplicativo} $8^5$ situa\c c\~oes diferentes, mas as que todas as pessoas saem em andares diferentes ocorrem do seguinte modo: a primeira tem 8 escolhas, a segunda apenas 7, pois n\~ao pode sair no mesmo andar da primeira, a terceira 6, a quarta 5 e a quinta 4, ou seja s\~ao $8.7.6.5.4$ casos favor\'aveis. Portanto a probabilidade deles ocorrerem \'e $$P_1= \frac{8.7.6.5.4}{8^5}=\frac{7.6.5.4}{8^4}=\frac{7.5.3}{8^3}=\frac{105}{512}$$

\item[(b)] A probabilidade de mais de uma pessoa descerem num mesmo andar \'e a probabilidade complementar do item anterior, ou seja: $$P_2=1-P_1=1-\frac{105}{512}=\frac{407}{512}$$

\end{enumerate}

\subsection{\color{red} Quest\~ao 2}

\begin{enumerate}

\item[(a)] Hip\'otese do {\it Princ\'\i pio da Indu\c c\~ao}: $a_1=a$; $\displaystyle a_{n+1}  = \frac{4a_n}{2+a_n^2}, \rm{para } n\geq 1$ e $0<a<\sqrt{2}$ e a tese \'e: $a_n<\sqrt{2}, \forall n\geq 1$

\item[(b)] Se $\displaystyle s=\frac{4a}{2+a^2}$ e pela hip\'otese de indu\c c\~ao $a>0$, ent\~ao $4a>0$ e $2+a^2>0$, portanto $s>0$

\item[(c)] Como $\displaystyle s=\frac{4a}{2+a^2}$ temos que: $$s^2=\frac{16a^2}{(2+a^2)^2}=\frac{16a^2}{4+4a^2+a^4}=\frac{16a^2}{(a^2-2)^2+8a^2}<\frac{16a^2}{8a^2}=2$$ portanto provamos que $s^2<2$.

\item[(d)] Temos que $s$ \'e sempre positiva e $0<s^2<2$, portanto se extrairmos a raiz quadrada obtemos: $0<s<\sqrt{2}$

\item[(e)] Como temos $a_n<\sqrt 2$ e $s=\displaystyle \frac{4a}{2+a^2}<\sqrt 2, \forall a, a<\sqrt 2$, logo: $a_{n+1}=\displaystyle \frac{4a_n}{2+a_n^2}<\sqrt 2<2$

\item[(f)] Para $n=1$ temos: $a_2=s<\sqrt 2$, \'e valida a hip\'otese. E como foi mostrado no item anterior: $a_{n+1}=\displaystyle \frac{4a_n}{2+a_n^2}<\sqrt 2$, assim conclu\'\i mos a indu\c c\~ao.

\end{enumerate}

\subsection{\color{red} Quest\~ao 3}

\begin{enumerate}

\item[(a)] Se $f$ \'e uma fun\c c\~ao cont\'\i nua em um intervalo $[a,b]$, ent\~ao o {\it Teorema do Valor Intermedi\'ario} diz que para todo $f(a)\leq k \leq f(b)$ existe um n\'umero $c\in (a,b)$ tal que: $f(k)=c$. 

\item[(b)] Considerando que a velocidade do corredor brasileiro possa ser expressa por uma fun\c c\~ao cont\'\i nua, $15 (km)= 15000 (m)$ e como ele percorreu este percurso em $44 (min)= 2640 (s)$ e $7 (seg)$, ou seja $2647 (seg)$, sua velocidade m\'edia foi $\displaystyle v_m= \frac{15000}{2647}\approx 5,6 (m/s)$. Como os corredores iniciam a corrida parados, temos que $v_0=0$ e considerando que ele tenha parado no instante que terminou a corrida, temos $v_{2647}=0$. Pelo teorema enunciado existe um \'unico momento $t$ em que $v_t=5,6 (m/s)$, mas como $5<5,6$ e $v_0=v_{2647}=0$, ent\~ao existem pelo menos dois instantes $a$ e $b$, por exemplo, em que a velocidade foi $5 (m/s)$.

\item[(c)] Qualquer situa\c c\~ao problema que pode ser modelada por uma fun\c c\~ao injetora.

\end{enumerate}

\chapter{ENADE 2014}

\section{\color{blue} Quest\~oes}

\subsection{\color{blue} Quest\~ao 1}

Os principais efeitos visuais da computa\c c\~ao gr\'afica vistos em uma tela s\~ao resultados de aplica\c c\~oes de transforma\c c\~oes lineares. Transla\c c\~ao, rota\c c\~ao, redimensionamento e altera\c c\~ao de cores s\~ao apenas alguns exemplos.

Considere que uma tela \'e cortada por dois eixos, $x$ e $y$, ortogonais entre si, formando um sistema de coordenadas com origem no centro da tela. Suponha que, nessa tela plana, existe a imagem de uma elipse com eixo maior de tamanho 4, paralelo ao eixo $x$, e cujos focos t\^em coordenadas $(-1,2)$ e $(1,2)$. Considere $T$ um operador linear definido em $\mathbb R^2$.

De acordo com as informa\c c\~oes acima, fa\c ca o que se pede nos itens a seguir, apresentando os c\'alculos utilizados na sua resolu\c c\~ao.

\begin{enumerate}

\item[(a)] Mostre que o ponto $(0,2+\sqrt{3})$ pertence \`a elipse. (valor: 3,0 pontos)

\item[(b)] Suponha que, em cada ponto da tela, seja aplicado o operador linear $T(x,y)=(x+y,-2x+4y)$. Quais ser\~ao as coordenadas dos focos da elipse ap\'os a aplica\c c\~ao de $T$? (valor: 3,0 pontos)

\item[(c)] Calcule os autovalores do operador linear $T(x,y)=(x+y,-2x+4y)$. (valor: 4,0 pontos)

\end{enumerate}
\subsection{\color{blue} Quest\~ao 2}

O n\'umero de ouro \'e conhecido h\'a mais de dois mil anos, sendo encontrado nas artes, nas pir\^amides do Egito e na natureza. Para construir o n\'umero de ouro apenas com o aux\'ilio de uma r\'egua n\~ao graduada e de um compasso, utiliza-se o seguinte procedimento: dado um segmento $AB$ qualquer, marca-se o seu ponto m\'edio; constr\'oi-se o segmento $BC$ perpendicular a $AB$ e com a metade do comprimento de $AB$; marca-se o ponto $E$ sobre a hipotenusa do tri\^angulo $ABC$, tal que $\overline{EC}$ e $\overline{BC}$ sejam iguais; e determina-se o ponto $D$ no segmento $AB$ tal que $\overline{AD}$ e $\overline{AE}$ sejam iguais. Com esse procedimento, o ponto $D$ divide o segmento $AB$ na raz\~ao \'aurea.

A partir da constru\c c\~ao geom\'etrica do n\'umero de ouro e considerando $x$ como o comprimento do segmento $AB$, fa\c ca o que se pede nos itens a seguir, apresentando os c\'alculos utilizados na sua resolu\c c\~ao.

\begin{enumerate}

\item[(a)] Determine o comprimento do segmento $AC$ em fun\c c\~ao de $x$. (valor: 4,0 pontos)

\item[(b)] Determine o comprimento do segmento $AD$ em fun\c c\~ao de $x$. (valor: 4,0 pontos)

\item[(c)] Determine o n\'umero de ouro dado pelo quociente $\overline{AB}\over\overline{AD}$. (valor: 2,0 pontos)
\end{enumerate}

\subsection{\color{blue} Quest\~ao 3}

A Torre de Han\'oi foi inventada por Edouard Lucas em 1883. H\'a uma hist\'oria sobre a Torre, imaginada pelo pr\'oprio Lucas:

No come\c co dos tempos, Deus criou a Torre de Brahma, que cont\'em tr\^es pinos de diamante e colocou no primeiro pino 64 discos de ouro maci\c co. Deus, ent\~ao, chamou seus sacerdotes e ordenou-lhes que transferissem todos os discos para o terceiro pino, segundo certas regras. Os sacerdotes, ent\~ao, obedeceram e com\c caram o seu trabalho, dia e noite. Quando eles terminassem, a torre de Brahma iria ruir e o mundo acabaria.

\vglue 10 pt

\pdfximage width 100pt height 110pt {hanoi.pdf}

\centerline{\pdfrefximage\pdflastximage}

Esse \'e um dos quebra-cabe\c cas matem\'aticos mais populares, que consiste de $n$ discos com um furo em seu centro e de tamanhos diferentes e de uma base com tr\^es pinos na posi\c c\~ao vertical onde s\~ao colocados os discos. O jogo mais simples \'e constituido de tr\^es pinos mas e quantidade pode variar, deixando o jogo mais dif\'icil \`a medida que o n\'umero de discos aumenta. Os discos formam uma torre onde todos s\~ao colocados em um dos pinos em ordem decrescente de tamanho. O objetivo do quebra-cabe\c ca \'e transferir toda a torre de discos para um dos outros pinos, que est\~ao inicialmente vazios, de modo que cada movimento \'e feito somente com um disco, nunca havendo um disco maior sobre um disco menor, como mostra a figura acima.

Considerando uma torre de Han\'oi de 3 pinos, fa\c ca o que se pede nos itens a seguir.

\begin{enumerate}

\item[(a)] Ao planejar uma aula de matem\'atica utilizando-se a Torre de Han\'oi, quais seriam os objetivos a serem alcan\c cados de acordo com os Par\^ametros Curriculares Nacionais e o que se espera com o uso de jogos no processo de ensino-aprendizagem? (valor: 3,0 pontos)

\item[(b)] Cite tr\^es conceitos matem\'aticos de Educa\c c\~ao B\'asica que podem ser explorados em sala de aula utilizando-se a Torre de Han\'oi? (valor: 3,0 pontos)

\item[(c)] Obtenha uma f\'ormula para o n\'umero m\'inimo de movimentos necess\'arios para resolver a Torre de Han\'oi com discos. Justifique a sua resposta. (valor: 4,0 pontos)

\end{enumerate}

\subsection{\color{blue} Quest\~ao }

Atualmente, a maioria dos editores de texto oferece o recurso de corre\c c\~ao ortogr\'afica. Esse recurso consiste em destacas, entre as palavras digitadas, aquelas com poss\'\i veis erros de grafia. Por exemplo, quando se digita a palavra ``caza'', o recurso de corre\c c\~ao destaca essa palavra, pois a palavra ``caza'' n\~ao existe na l\'\i ngua portuguesa. Tamb\'em \'e comum o recurso de corre\c c\~ao ortogr\'afica sugerir uma outra palavra para substituir a palavra incorreta.

A sugest\~ao de quais palavras podem substituir a palavra incorreta \'e feita com uma medida da dist\^ancia entre a palavra incorreta e as palavras que constam no dicion\'ario do editor de texto. Existem diversas maneiras de medir a dist\^ancia entre duas palavras. Uma delas \'e a denominada {\it Dist\^ancia de Hamming}, na qual a medida da dist\^ancia entre duas palavras $x$ e $y$, em suas respectivas posi\c c\~oes. Mais formalmente, se $x=x_1x_2x_3\ldots x_n$ e $y=y_1y_2y_3\ldots y_n$ s\~ao palavras em que $x_i$ e $y_i$ s\~ao letras do alfabeto, para $i=1,2,3,\ldots,n$, ent\~ao $d(x,y)=\#(\{i:x_i\neq y_i$, com $i=1,2,3,\ldots n\})$, em que $\#(\{3\})=1$, j\'a que elas diferem apenas na terceira letra.

A partir dessas informa\c c\~oes, fa\c ca o que se pede nos itens a seguir.

\begin{enumerate}

\item[(a)] Mostre que a Dist\^ancia de Hamming \'e uma m\'etrica no conjunto das palavras com letras. (valor: 5,0 pontos)

\item[(b)] Mostre que o conjunto das palavras com letras, munido da Dist\^ancia de Hamming, \'e um espa\c co m\'etrico discreto. (valor: 5,0 pontos)

\end{enumerate}

\subsection{\color{blue} Quest\~ao 3}

Uma equa\c c\~ao diofantina linear nas inc\'ognitas $x$ e $y$ \'e uma equa\c c\~ao da forma $ax+by=c$, em que $a$, $b$ e $c$ s\~ao inteiros, e as \'unicas solu\c c\~oes $(x_0,y_0)$ que interessam s\~ao aquelas em que $x_0, y_0 \in \mathbb Z$.

Nesse contexto, considere que os ingressos de um cinema custam R\$ 9,00 para estudantes e R\$ 15,00 para o p\'ublico geral, e que, em certo dia, durante determinado per\'\i odo, a arrecada\c c\~ao nas bilheterias desse cinema foi R\$ 246,00.

A partir das informa\c c\~oes acima, fa\c ca o que se pede nos itens a seguir.

\begin{enumerate}

\item[(a)] Obtenha ema equa\c c\~ao diofantina linear que modele a situa\c c\~ao acima, indicando o significado das inc\'ognitas. (valor: 3,0 pontos)

\item[(b)] Quantas e quais s\~ao as solu\c c\~oes do problema descrito no item (a)? (valor: 7,0 pontos)

\end{enumerate}

\section{\color{red} Solu\c c\~oes}

\subsection{\color{red} Quest\~ao 1}

\begin{enumerate}

\item[(a)]

\item[(b)]

\end{enumerate}

\part{}

\end{document}